%% Main document for Source of India Constitution
%% Work in progress - currently includes selected parts and schedules

\documentclass[a4paper,showamendments]{soi}

% Load configuration settings
%% Configuration file for Source of India LaTeX class
%% This file contains customizable settings and preferences

%% Amendment display preferences
% Uncomment the following to hide amendments globally
% \@soishowamendmentsfalse

%% Typography fine-tuning
\setlength{\parskip}{\baselineskip}
\setlength{\parindent}{0pt}

%% Footnote spacing adjustments
\setlength{\footnotesep}{0.75em}
\addtolength{\skip\footins}{0.25em}

%% Enhanced spacing for constitutional elements
\newlength{\articlespacing}
\setlength{\articlespacing}{1.5\baselineskip}

\newlength{\clausespacing}
\setlength{\clausespacing}{0.75\baselineskip}

%% Custom formatting for special constitutional terms
\providecommand{\soiConstitutionTerm}[1]{\textsc{#1}}
\providecommand{\soiActCitation}[1]{\textit{#1}}

%% Header customization
\providecommand{\soiHeaderFont}{\small\itshape}

%% Page layout fine adjustments
\addtolength{\textheight}{0.5cm}
\addtolength{\textwidth}{0.25cm}

%% Constitutional document metadata
\providecommand{\soiConstitutionTitle}{THE CONSTITUTION OF INDIA}
\providecommand{\soiConstitutionSubtitle}{As amended up to the One Hundred and Fifth Amendment Act, 2021}
\providecommand{\soiConstitutionDate}{26th November, 1949}

%% Amendment citation formatting
\providecommand{\soiAmendmentCite}[2]{%
    Constitution (#1 Amendment) Act, #2%
}

%% Effective date formatting
\providecommand{\soiEffectiveDate}[1]{%
    \textit{w.e.f.} #1%
}

%% Development mode settings
\if@soidraftmode
    % Show additional information in draft mode
    \fancyfoot[LE]{\tiny Draft Mode}
    \fancyfoot[RO]{\tiny \today}
\fi

%% Conditional compilation settings
% Uncomment to compile only specific parts
% \newif\ifcompilepart@i\compilepart@itrue
% \newif\ifcompilepart@iii\compilepart@iiitrue
% \newif\ifcompilepart@iv\compilepart@ivfalse

%% Table of contents depth control
\setcounter{tocdepth}{3}

%% Cross-reference formatting
\providecommand{\soiArtRef}[1]{Article~\ref{art:#1}}
\providecommand{\soiPartRef}[1]{Part~\ref{part:#1}}
\providecommand{\soiSchedRef}[1]{Schedule~\ref{sched:#1}}

%% Special constitutional formatting
\providecommand{\soiPreambleFormat}{\centering\large\itshape}
\providecommand{\soiSignatureFormat}{\raggedleft\normalsize}

%% End of configuration

\begin{document}

% Title page and preliminaries
\begin{titlepage}
    \centering
    \vspace*{2cm}
    
    {\Huge\bfseries THE CONSTITUTION OF INDIA}
    
    \vspace{1cm}
    {\Large As amended up to the One Hundred and Fifth Amendment Act, 2021}
    
    \vspace{2cm}
    {\large Adopted by the Constituent Assembly on 26th November, 1949}
    
    \vfill
    
    {\large Government of India\\Ministry of Law and Justice}
    
    \vspace{1cm}
    {\normalsize Source of India LaTeX Project - Work in Progress}
\end{titlepage}

% Table of contents
%% Table of Contents for Source of India Constitution
%% Can be compiled independently from any directory level

% Path-aware configuration loading
\@ifundefined{documentclass}{
    \documentclass[a4paper,showamendments]{soi}
    \IfFileExists{config.tex}{%% Configuration file for Source of India LaTeX class
%% This file contains customizable settings and preferences

%% Amendment display preferences
% Uncomment the following to hide amendments globally
% \@soishowamendmentsfalse

%% Typography fine-tuning
\setlength{\parskip}{\baselineskip}
\setlength{\parindent}{0pt}

%% Footnote spacing adjustments
\setlength{\footnotesep}{0.75em}
\addtolength{\skip\footins}{0.25em}

%% Enhanced spacing for constitutional elements
\newlength{\articlespacing}
\setlength{\articlespacing}{1.5\baselineskip}

\newlength{\clausespacing}
\setlength{\clausespacing}{0.75\baselineskip}

%% Custom formatting for special constitutional terms
\providecommand{\soiConstitutionTerm}[1]{\textsc{#1}}
\providecommand{\soiActCitation}[1]{\textit{#1}}

%% Header customization
\providecommand{\soiHeaderFont}{\small\itshape}

%% Page layout fine adjustments
\addtolength{\textheight}{0.5cm}
\addtolength{\textwidth}{0.25cm}

%% Constitutional document metadata
\providecommand{\soiConstitutionTitle}{THE CONSTITUTION OF INDIA}
\providecommand{\soiConstitutionSubtitle}{As amended up to the One Hundred and Fifth Amendment Act, 2021}
\providecommand{\soiConstitutionDate}{26th November, 1949}

%% Amendment citation formatting
\providecommand{\soiAmendmentCite}[2]{%
    Constitution (#1 Amendment) Act, #2%
}

%% Effective date formatting
\providecommand{\soiEffectiveDate}[1]{%
    \textit{w.e.f.} #1%
}

%% Development mode settings
\if@soidraftmode
    % Show additional information in draft mode
    \fancyfoot[LE]{\tiny Draft Mode}
    \fancyfoot[RO]{\tiny \today}
\fi

%% Conditional compilation settings
% Uncomment to compile only specific parts
% \newif\ifcompilepart@i\compilepart@itrue
% \newif\ifcompilepart@iii\compilepart@iiitrue
% \newif\ifcompilepart@iv\compilepart@ivfalse

%% Table of contents depth control
\setcounter{tocdepth}{3}

%% Cross-reference formatting
\providecommand{\soiArtRef}[1]{Article~\ref{art:#1}}
\providecommand{\soiPartRef}[1]{Part~\ref{part:#1}}
\providecommand{\soiSchedRef}[1]{Schedule~\ref{sched:#1}}

%% Special constitutional formatting
\providecommand{\soiPreambleFormat}{\centering\large\itshape}
\providecommand{\soiSignatureFormat}{\raggedleft\normalsize}

%% End of configuration}{\IfFileExists{../config.tex}{%% Configuration file for Source of India LaTeX class
%% This file contains customizable settings and preferences

%% Amendment display preferences
% Uncomment the following to hide amendments globally
% \@soishowamendmentsfalse

%% Typography fine-tuning
\setlength{\parskip}{\baselineskip}
\setlength{\parindent}{0pt}

%% Footnote spacing adjustments
\setlength{\footnotesep}{0.75em}
\addtolength{\skip\footins}{0.25em}

%% Enhanced spacing for constitutional elements
\newlength{\articlespacing}
\setlength{\articlespacing}{1.5\baselineskip}

\newlength{\clausespacing}
\setlength{\clausespacing}{0.75\baselineskip}

%% Custom formatting for special constitutional terms
\providecommand{\soiConstitutionTerm}[1]{\textsc{#1}}
\providecommand{\soiActCitation}[1]{\textit{#1}}

%% Header customization
\providecommand{\soiHeaderFont}{\small\itshape}

%% Page layout fine adjustments
\addtolength{\textheight}{0.5cm}
\addtolength{\textwidth}{0.25cm}

%% Constitutional document metadata
\providecommand{\soiConstitutionTitle}{THE CONSTITUTION OF INDIA}
\providecommand{\soiConstitutionSubtitle}{As amended up to the One Hundred and Fifth Amendment Act, 2021}
\providecommand{\soiConstitutionDate}{26th November, 1949}

%% Amendment citation formatting
\providecommand{\soiAmendmentCite}[2]{%
    Constitution (#1 Amendment) Act, #2%
}

%% Effective date formatting
\providecommand{\soiEffectiveDate}[1]{%
    \textit{w.e.f.} #1%
}

%% Development mode settings
\if@soidraftmode
    % Show additional information in draft mode
    \fancyfoot[LE]{\tiny Draft Mode}
    \fancyfoot[RO]{\tiny \today}
\fi

%% Conditional compilation settings
% Uncomment to compile only specific parts
% \newif\ifcompilepart@i\compilepart@itrue
% \newif\ifcompilepart@iii\compilepart@iiitrue
% \newif\ifcompilepart@iv\compilepart@ivfalse

%% Table of contents depth control
\setcounter{tocdepth}{3}

%% Cross-reference formatting
\providecommand{\soiArtRef}[1]{Article~\ref{art:#1}}
\providecommand{\soiPartRef}[1]{Part~\ref{part:#1}}
\providecommand{\soiSchedRef}[1]{Schedule~\ref{sched:#1}}

%% Special constitutional formatting
\providecommand{\soiPreambleFormat}{\centering\large\itshape}
\providecommand{\soiSignatureFormat}{\raggedleft\normalsize}

%% End of configuration}{\IfFileExists{../../config.tex}{%% Configuration file for Source of India LaTeX class
%% This file contains customizable settings and preferences

%% Amendment display preferences
% Uncomment the following to hide amendments globally
% \@soishowamendmentsfalse

%% Typography fine-tuning
\setlength{\parskip}{\baselineskip}
\setlength{\parindent}{0pt}

%% Footnote spacing adjustments
\setlength{\footnotesep}{0.75em}
\addtolength{\skip\footins}{0.25em}

%% Enhanced spacing for constitutional elements
\newlength{\articlespacing}
\setlength{\articlespacing}{1.5\baselineskip}

\newlength{\clausespacing}
\setlength{\clausespacing}{0.75\baselineskip}

%% Custom formatting for special constitutional terms
\providecommand{\soiConstitutionTerm}[1]{\textsc{#1}}
\providecommand{\soiActCitation}[1]{\textit{#1}}

%% Header customization
\providecommand{\soiHeaderFont}{\small\itshape}

%% Page layout fine adjustments
\addtolength{\textheight}{0.5cm}
\addtolength{\textwidth}{0.25cm}

%% Constitutional document metadata
\providecommand{\soiConstitutionTitle}{THE CONSTITUTION OF INDIA}
\providecommand{\soiConstitutionSubtitle}{As amended up to the One Hundred and Fifth Amendment Act, 2021}
\providecommand{\soiConstitutionDate}{26th November, 1949}

%% Amendment citation formatting
\providecommand{\soiAmendmentCite}[2]{%
    Constitution (#1 Amendment) Act, #2%
}

%% Effective date formatting
\providecommand{\soiEffectiveDate}[1]{%
    \textit{w.e.f.} #1%
}

%% Development mode settings
\if@soidraftmode
    % Show additional information in draft mode
    \fancyfoot[LE]{\tiny Draft Mode}
    \fancyfoot[RO]{\tiny \today}
\fi

%% Conditional compilation settings
% Uncomment to compile only specific parts
% \newif\ifcompilepart@i\compilepart@itrue
% \newif\ifcompilepart@iii\compilepart@iiitrue
% \newif\ifcompilepart@iv\compilepart@ivfalse

%% Table of contents depth control
\setcounter{tocdepth}{3}

%% Cross-reference formatting
\providecommand{\soiArtRef}[1]{Article~\ref{art:#1}}
\providecommand{\soiPartRef}[1]{Part~\ref{part:#1}}
\providecommand{\soiSchedRef}[1]{Schedule~\ref{sched:#1}}

%% Special constitutional formatting
\providecommand{\soiPreambleFormat}{\centering\large\itshape}
\providecommand{\soiSignatureFormat}{\raggedleft\normalsize}

%% End of configuration}{}}}
    \begin{document}
}{}

\clearpage
\phantomsection
\addcontentsline{toc}{section}{TABLE OF CONTENTS}

\begin{center}
    {\LARGE\bfseries TABLE OF CONTENTS}
\end{center}

\vspace{2\baselineskip}

\tableofcontents

\clearpage

\@ifundefined{documentclass}{}{\end{document}}

% Part I: The Union and Its Territory
%% Part I: The Union and Its Territory
%% Can be compiled independently from any directory level

% Check if this is being compiled as a standalone document
\ifx\documentclass\undefined\else
    % Standalone compilation
    \documentclass[a4paper,showamendments]{soi}
    
    % Path-aware config loading
    \IfFileExists{config.tex}{%% Configuration file for Source of India LaTeX class
%% This file contains customizable settings and preferences

%% Amendment display preferences
% Uncomment the following to hide amendments globally
% \@soishowamendmentsfalse

%% Typography fine-tuning
\setlength{\parskip}{\baselineskip}
\setlength{\parindent}{0pt}

%% Footnote spacing adjustments
\setlength{\footnotesep}{0.75em}
\addtolength{\skip\footins}{0.25em}

%% Enhanced spacing for constitutional elements
\newlength{\articlespacing}
\setlength{\articlespacing}{1.5\baselineskip}

\newlength{\clausespacing}
\setlength{\clausespacing}{0.75\baselineskip}

%% Custom formatting for special constitutional terms
\providecommand{\soiConstitutionTerm}[1]{\textsc{#1}}
\providecommand{\soiActCitation}[1]{\textit{#1}}

%% Header customization
\providecommand{\soiHeaderFont}{\small\itshape}

%% Page layout fine adjustments
\addtolength{\textheight}{0.5cm}
\addtolength{\textwidth}{0.25cm}

%% Constitutional document metadata
\providecommand{\soiConstitutionTitle}{THE CONSTITUTION OF INDIA}
\providecommand{\soiConstitutionSubtitle}{As amended up to the One Hundred and Fifth Amendment Act, 2021}
\providecommand{\soiConstitutionDate}{26th November, 1949}

%% Amendment citation formatting
\providecommand{\soiAmendmentCite}[2]{%
    Constitution (#1 Amendment) Act, #2%
}

%% Effective date formatting
\providecommand{\soiEffectiveDate}[1]{%
    \textit{w.e.f.} #1%
}

%% Development mode settings
\if@soidraftmode
    % Show additional information in draft mode
    \fancyfoot[LE]{\tiny Draft Mode}
    \fancyfoot[RO]{\tiny \today}
\fi

%% Conditional compilation settings
% Uncomment to compile only specific parts
% \newif\ifcompilepart@i\compilepart@itrue
% \newif\ifcompilepart@iii\compilepart@iiitrue
% \newif\ifcompilepart@iv\compilepart@ivfalse

%% Table of contents depth control
\setcounter{tocdepth}{3}

%% Cross-reference formatting
\providecommand{\soiArtRef}[1]{Article~\ref{art:#1}}
\providecommand{\soiPartRef}[1]{Part~\ref{part:#1}}
\providecommand{\soiSchedRef}[1]{Schedule~\ref{sched:#1}}

%% Special constitutional formatting
\providecommand{\soiPreambleFormat}{\centering\large\itshape}
\providecommand{\soiSignatureFormat}{\raggedleft\normalsize}

%% End of configuration}{%
        \IfFileExists{../config.tex}{%% Configuration file for Source of India LaTeX class
%% This file contains customizable settings and preferences

%% Amendment display preferences
% Uncomment the following to hide amendments globally
% \@soishowamendmentsfalse

%% Typography fine-tuning
\setlength{\parskip}{\baselineskip}
\setlength{\parindent}{0pt}

%% Footnote spacing adjustments
\setlength{\footnotesep}{0.75em}
\addtolength{\skip\footins}{0.25em}

%% Enhanced spacing for constitutional elements
\newlength{\articlespacing}
\setlength{\articlespacing}{1.5\baselineskip}

\newlength{\clausespacing}
\setlength{\clausespacing}{0.75\baselineskip}

%% Custom formatting for special constitutional terms
\providecommand{\soiConstitutionTerm}[1]{\textsc{#1}}
\providecommand{\soiActCitation}[1]{\textit{#1}}

%% Header customization
\providecommand{\soiHeaderFont}{\small\itshape}

%% Page layout fine adjustments
\addtolength{\textheight}{0.5cm}
\addtolength{\textwidth}{0.25cm}

%% Constitutional document metadata
\providecommand{\soiConstitutionTitle}{THE CONSTITUTION OF INDIA}
\providecommand{\soiConstitutionSubtitle}{As amended up to the One Hundred and Fifth Amendment Act, 2021}
\providecommand{\soiConstitutionDate}{26th November, 1949}

%% Amendment citation formatting
\providecommand{\soiAmendmentCite}[2]{%
    Constitution (#1 Amendment) Act, #2%
}

%% Effective date formatting
\providecommand{\soiEffectiveDate}[1]{%
    \textit{w.e.f.} #1%
}

%% Development mode settings
\if@soidraftmode
    % Show additional information in draft mode
    \fancyfoot[LE]{\tiny Draft Mode}
    \fancyfoot[RO]{\tiny \today}
\fi

%% Conditional compilation settings
% Uncomment to compile only specific parts
% \newif\ifcompilepart@i\compilepart@itrue
% \newif\ifcompilepart@iii\compilepart@iiitrue
% \newif\ifcompilepart@iv\compilepart@ivfalse

%% Table of contents depth control
\setcounter{tocdepth}{3}

%% Cross-reference formatting
\providecommand{\soiArtRef}[1]{Article~\ref{art:#1}}
\providecommand{\soiPartRef}[1]{Part~\ref{part:#1}}
\providecommand{\soiSchedRef}[1]{Schedule~\ref{sched:#1}}

%% Special constitutional formatting
\providecommand{\soiPreambleFormat}{\centering\large\itshape}
\providecommand{\soiSignatureFormat}{\raggedleft\normalsize}

%% End of configuration}{%
            \IfFileExists{../../config.tex}{%% Configuration file for Source of India LaTeX class
%% This file contains customizable settings and preferences

%% Amendment display preferences
% Uncomment the following to hide amendments globally
% \@soishowamendmentsfalse

%% Typography fine-tuning
\setlength{\parskip}{\baselineskip}
\setlength{\parindent}{0pt}

%% Footnote spacing adjustments
\setlength{\footnotesep}{0.75em}
\addtolength{\skip\footins}{0.25em}

%% Enhanced spacing for constitutional elements
\newlength{\articlespacing}
\setlength{\articlespacing}{1.5\baselineskip}

\newlength{\clausespacing}
\setlength{\clausespacing}{0.75\baselineskip}

%% Custom formatting for special constitutional terms
\providecommand{\soiConstitutionTerm}[1]{\textsc{#1}}
\providecommand{\soiActCitation}[1]{\textit{#1}}

%% Header customization
\providecommand{\soiHeaderFont}{\small\itshape}

%% Page layout fine adjustments
\addtolength{\textheight}{0.5cm}
\addtolength{\textwidth}{0.25cm}

%% Constitutional document metadata
\providecommand{\soiConstitutionTitle}{THE CONSTITUTION OF INDIA}
\providecommand{\soiConstitutionSubtitle}{As amended up to the One Hundred and Fifth Amendment Act, 2021}
\providecommand{\soiConstitutionDate}{26th November, 1949}

%% Amendment citation formatting
\providecommand{\soiAmendmentCite}[2]{%
    Constitution (#1 Amendment) Act, #2%
}

%% Effective date formatting
\providecommand{\soiEffectiveDate}[1]{%
    \textit{w.e.f.} #1%
}

%% Development mode settings
\if@soidraftmode
    % Show additional information in draft mode
    \fancyfoot[LE]{\tiny Draft Mode}
    \fancyfoot[RO]{\tiny \today}
\fi

%% Conditional compilation settings
% Uncomment to compile only specific parts
% \newif\ifcompilepart@i\compilepart@itrue
% \newif\ifcompilepart@iii\compilepart@iiitrue
% \newif\ifcompilepart@iv\compilepart@ivfalse

%% Table of contents depth control
\setcounter{tocdepth}{3}

%% Cross-reference formatting
\providecommand{\soiArtRef}[1]{Article~\ref{art:#1}}
\providecommand{\soiPartRef}[1]{Part~\ref{part:#1}}
\providecommand{\soiSchedRef}[1]{Schedule~\ref{sched:#1}}

%% Special constitutional formatting
\providecommand{\soiPreambleFormat}{\centering\large\itshape}
\providecommand{\soiSignatureFormat}{\raggedleft\normalsize}

%% End of configuration}{%
                % Use default config if none found
            }%
        }%
    }
    
    \begin{document}
    \def\soistandalone{true}
\fi

\soiPart{I}{THE UNION AND ITS TERRITORY}

% Path-aware article inclusion - try different relative paths
\IfFileExists{../articles/article_001.tex}{%% Article 1: Name and territory of the Union
%% Can be compiled independently from any directory level

% Check if this is being compiled as a standalone document
\ifx\documentclass\undefined\else
    % Standalone compilation
    \documentclass[a4paper,showamendments]{soi}
    
    % Path-aware config loading
    \IfFileExists{config.tex}{%% Configuration file for Source of India LaTeX class
%% This file contains customizable settings and preferences

%% Amendment display preferences
% Uncomment the following to hide amendments globally
% \@soishowamendmentsfalse

%% Typography fine-tuning
\setlength{\parskip}{\baselineskip}
\setlength{\parindent}{0pt}

%% Footnote spacing adjustments
\setlength{\footnotesep}{0.75em}
\addtolength{\skip\footins}{0.25em}

%% Enhanced spacing for constitutional elements
\newlength{\articlespacing}
\setlength{\articlespacing}{1.5\baselineskip}

\newlength{\clausespacing}
\setlength{\clausespacing}{0.75\baselineskip}

%% Custom formatting for special constitutional terms
\providecommand{\soiConstitutionTerm}[1]{\textsc{#1}}
\providecommand{\soiActCitation}[1]{\textit{#1}}

%% Header customization
\providecommand{\soiHeaderFont}{\small\itshape}

%% Page layout fine adjustments
\addtolength{\textheight}{0.5cm}
\addtolength{\textwidth}{0.25cm}

%% Constitutional document metadata
\providecommand{\soiConstitutionTitle}{THE CONSTITUTION OF INDIA}
\providecommand{\soiConstitutionSubtitle}{As amended up to the One Hundred and Fifth Amendment Act, 2021}
\providecommand{\soiConstitutionDate}{26th November, 1949}

%% Amendment citation formatting
\providecommand{\soiAmendmentCite}[2]{%
    Constitution (#1 Amendment) Act, #2%
}

%% Effective date formatting
\providecommand{\soiEffectiveDate}[1]{%
    \textit{w.e.f.} #1%
}

%% Development mode settings
\if@soidraftmode
    % Show additional information in draft mode
    \fancyfoot[LE]{\tiny Draft Mode}
    \fancyfoot[RO]{\tiny \today}
\fi

%% Conditional compilation settings
% Uncomment to compile only specific parts
% \newif\ifcompilepart@i\compilepart@itrue
% \newif\ifcompilepart@iii\compilepart@iiitrue
% \newif\ifcompilepart@iv\compilepart@ivfalse

%% Table of contents depth control
\setcounter{tocdepth}{3}

%% Cross-reference formatting
\providecommand{\soiArtRef}[1]{Article~\ref{art:#1}}
\providecommand{\soiPartRef}[1]{Part~\ref{part:#1}}
\providecommand{\soiSchedRef}[1]{Schedule~\ref{sched:#1}}

%% Special constitutional formatting
\providecommand{\soiPreambleFormat}{\centering\large\itshape}
\providecommand{\soiSignatureFormat}{\raggedleft\normalsize}

%% End of configuration}{%
        \IfFileExists{../config.tex}{%% Configuration file for Source of India LaTeX class
%% This file contains customizable settings and preferences

%% Amendment display preferences
% Uncomment the following to hide amendments globally
% \@soishowamendmentsfalse

%% Typography fine-tuning
\setlength{\parskip}{\baselineskip}
\setlength{\parindent}{0pt}

%% Footnote spacing adjustments
\setlength{\footnotesep}{0.75em}
\addtolength{\skip\footins}{0.25em}

%% Enhanced spacing for constitutional elements
\newlength{\articlespacing}
\setlength{\articlespacing}{1.5\baselineskip}

\newlength{\clausespacing}
\setlength{\clausespacing}{0.75\baselineskip}

%% Custom formatting for special constitutional terms
\providecommand{\soiConstitutionTerm}[1]{\textsc{#1}}
\providecommand{\soiActCitation}[1]{\textit{#1}}

%% Header customization
\providecommand{\soiHeaderFont}{\small\itshape}

%% Page layout fine adjustments
\addtolength{\textheight}{0.5cm}
\addtolength{\textwidth}{0.25cm}

%% Constitutional document metadata
\providecommand{\soiConstitutionTitle}{THE CONSTITUTION OF INDIA}
\providecommand{\soiConstitutionSubtitle}{As amended up to the One Hundred and Fifth Amendment Act, 2021}
\providecommand{\soiConstitutionDate}{26th November, 1949}

%% Amendment citation formatting
\providecommand{\soiAmendmentCite}[2]{%
    Constitution (#1 Amendment) Act, #2%
}

%% Effective date formatting
\providecommand{\soiEffectiveDate}[1]{%
    \textit{w.e.f.} #1%
}

%% Development mode settings
\if@soidraftmode
    % Show additional information in draft mode
    \fancyfoot[LE]{\tiny Draft Mode}
    \fancyfoot[RO]{\tiny \today}
\fi

%% Conditional compilation settings
% Uncomment to compile only specific parts
% \newif\ifcompilepart@i\compilepart@itrue
% \newif\ifcompilepart@iii\compilepart@iiitrue
% \newif\ifcompilepart@iv\compilepart@ivfalse

%% Table of contents depth control
\setcounter{tocdepth}{3}

%% Cross-reference formatting
\providecommand{\soiArtRef}[1]{Article~\ref{art:#1}}
\providecommand{\soiPartRef}[1]{Part~\ref{part:#1}}
\providecommand{\soiSchedRef}[1]{Schedule~\ref{sched:#1}}

%% Special constitutional formatting
\providecommand{\soiPreambleFormat}{\centering\large\itshape}
\providecommand{\soiSignatureFormat}{\raggedleft\normalsize}

%% End of configuration}{%
            \IfFileExists{../../config.tex}{%% Configuration file for Source of India LaTeX class
%% This file contains customizable settings and preferences

%% Amendment display preferences
% Uncomment the following to hide amendments globally
% \@soishowamendmentsfalse

%% Typography fine-tuning
\setlength{\parskip}{\baselineskip}
\setlength{\parindent}{0pt}

%% Footnote spacing adjustments
\setlength{\footnotesep}{0.75em}
\addtolength{\skip\footins}{0.25em}

%% Enhanced spacing for constitutional elements
\newlength{\articlespacing}
\setlength{\articlespacing}{1.5\baselineskip}

\newlength{\clausespacing}
\setlength{\clausespacing}{0.75\baselineskip}

%% Custom formatting for special constitutional terms
\providecommand{\soiConstitutionTerm}[1]{\textsc{#1}}
\providecommand{\soiActCitation}[1]{\textit{#1}}

%% Header customization
\providecommand{\soiHeaderFont}{\small\itshape}

%% Page layout fine adjustments
\addtolength{\textheight}{0.5cm}
\addtolength{\textwidth}{0.25cm}

%% Constitutional document metadata
\providecommand{\soiConstitutionTitle}{THE CONSTITUTION OF INDIA}
\providecommand{\soiConstitutionSubtitle}{As amended up to the One Hundred and Fifth Amendment Act, 2021}
\providecommand{\soiConstitutionDate}{26th November, 1949}

%% Amendment citation formatting
\providecommand{\soiAmendmentCite}[2]{%
    Constitution (#1 Amendment) Act, #2%
}

%% Effective date formatting
\providecommand{\soiEffectiveDate}[1]{%
    \textit{w.e.f.} #1%
}

%% Development mode settings
\if@soidraftmode
    % Show additional information in draft mode
    \fancyfoot[LE]{\tiny Draft Mode}
    \fancyfoot[RO]{\tiny \today}
\fi

%% Conditional compilation settings
% Uncomment to compile only specific parts
% \newif\ifcompilepart@i\compilepart@itrue
% \newif\ifcompilepart@iii\compilepart@iiitrue
% \newif\ifcompilepart@iv\compilepart@ivfalse

%% Table of contents depth control
\setcounter{tocdepth}{3}

%% Cross-reference formatting
\providecommand{\soiArtRef}[1]{Article~\ref{art:#1}}
\providecommand{\soiPartRef}[1]{Part~\ref{part:#1}}
\providecommand{\soiSchedRef}[1]{Schedule~\ref{sched:#1}}

%% Special constitutional formatting
\providecommand{\soiPreambleFormat}{\centering\large\itshape}
\providecommand{\soiSignatureFormat}{\raggedleft\normalsize}

%% End of configuration}{%
                % Use default config if none found
            }%
        }%
    }
    
    \begin{document}
    \def\soistandalone{true}
\fi

\soiArticle{Name and territory of the Union}{%
    \soiClause{India, that is Bharat, shall be a Union of States.}
    
    \soiClause{The States and the territories thereof shall be as specified in the \soiAmendment{First Schedule}\soiAmendmentData{Constitution (Seventh Amendment) Act, 1956, s. 2 \soiWef{1-11-1956}}.}
    
    \soiClause{The territory of India shall comprise—
        \soiSubClause{the territories of the States;}
        \soiSubClause{the Union territories specified in the First Schedule; and}
        \soiSubClause{such other territories as may be acquired.}
    }
}

\ifdefined\soistandalone\end{document}\fi}{%
    \IfFileExists{articles/article_001.tex}{%% Article 1: Name and territory of the Union
%% Can be compiled independently from any directory level

% Check if this is being compiled as a standalone document
\ifx\documentclass\undefined\else
    % Standalone compilation
    \documentclass[a4paper,showamendments]{soi}
    
    % Path-aware config loading
    \IfFileExists{config.tex}{%% Configuration file for Source of India LaTeX class
%% This file contains customizable settings and preferences

%% Amendment display preferences
% Uncomment the following to hide amendments globally
% \@soishowamendmentsfalse

%% Typography fine-tuning
\setlength{\parskip}{\baselineskip}
\setlength{\parindent}{0pt}

%% Footnote spacing adjustments
\setlength{\footnotesep}{0.75em}
\addtolength{\skip\footins}{0.25em}

%% Enhanced spacing for constitutional elements
\newlength{\articlespacing}
\setlength{\articlespacing}{1.5\baselineskip}

\newlength{\clausespacing}
\setlength{\clausespacing}{0.75\baselineskip}

%% Custom formatting for special constitutional terms
\providecommand{\soiConstitutionTerm}[1]{\textsc{#1}}
\providecommand{\soiActCitation}[1]{\textit{#1}}

%% Header customization
\providecommand{\soiHeaderFont}{\small\itshape}

%% Page layout fine adjustments
\addtolength{\textheight}{0.5cm}
\addtolength{\textwidth}{0.25cm}

%% Constitutional document metadata
\providecommand{\soiConstitutionTitle}{THE CONSTITUTION OF INDIA}
\providecommand{\soiConstitutionSubtitle}{As amended up to the One Hundred and Fifth Amendment Act, 2021}
\providecommand{\soiConstitutionDate}{26th November, 1949}

%% Amendment citation formatting
\providecommand{\soiAmendmentCite}[2]{%
    Constitution (#1 Amendment) Act, #2%
}

%% Effective date formatting
\providecommand{\soiEffectiveDate}[1]{%
    \textit{w.e.f.} #1%
}

%% Development mode settings
\if@soidraftmode
    % Show additional information in draft mode
    \fancyfoot[LE]{\tiny Draft Mode}
    \fancyfoot[RO]{\tiny \today}
\fi

%% Conditional compilation settings
% Uncomment to compile only specific parts
% \newif\ifcompilepart@i\compilepart@itrue
% \newif\ifcompilepart@iii\compilepart@iiitrue
% \newif\ifcompilepart@iv\compilepart@ivfalse

%% Table of contents depth control
\setcounter{tocdepth}{3}

%% Cross-reference formatting
\providecommand{\soiArtRef}[1]{Article~\ref{art:#1}}
\providecommand{\soiPartRef}[1]{Part~\ref{part:#1}}
\providecommand{\soiSchedRef}[1]{Schedule~\ref{sched:#1}}

%% Special constitutional formatting
\providecommand{\soiPreambleFormat}{\centering\large\itshape}
\providecommand{\soiSignatureFormat}{\raggedleft\normalsize}

%% End of configuration}{%
        \IfFileExists{../config.tex}{%% Configuration file for Source of India LaTeX class
%% This file contains customizable settings and preferences

%% Amendment display preferences
% Uncomment the following to hide amendments globally
% \@soishowamendmentsfalse

%% Typography fine-tuning
\setlength{\parskip}{\baselineskip}
\setlength{\parindent}{0pt}

%% Footnote spacing adjustments
\setlength{\footnotesep}{0.75em}
\addtolength{\skip\footins}{0.25em}

%% Enhanced spacing for constitutional elements
\newlength{\articlespacing}
\setlength{\articlespacing}{1.5\baselineskip}

\newlength{\clausespacing}
\setlength{\clausespacing}{0.75\baselineskip}

%% Custom formatting for special constitutional terms
\providecommand{\soiConstitutionTerm}[1]{\textsc{#1}}
\providecommand{\soiActCitation}[1]{\textit{#1}}

%% Header customization
\providecommand{\soiHeaderFont}{\small\itshape}

%% Page layout fine adjustments
\addtolength{\textheight}{0.5cm}
\addtolength{\textwidth}{0.25cm}

%% Constitutional document metadata
\providecommand{\soiConstitutionTitle}{THE CONSTITUTION OF INDIA}
\providecommand{\soiConstitutionSubtitle}{As amended up to the One Hundred and Fifth Amendment Act, 2021}
\providecommand{\soiConstitutionDate}{26th November, 1949}

%% Amendment citation formatting
\providecommand{\soiAmendmentCite}[2]{%
    Constitution (#1 Amendment) Act, #2%
}

%% Effective date formatting
\providecommand{\soiEffectiveDate}[1]{%
    \textit{w.e.f.} #1%
}

%% Development mode settings
\if@soidraftmode
    % Show additional information in draft mode
    \fancyfoot[LE]{\tiny Draft Mode}
    \fancyfoot[RO]{\tiny \today}
\fi

%% Conditional compilation settings
% Uncomment to compile only specific parts
% \newif\ifcompilepart@i\compilepart@itrue
% \newif\ifcompilepart@iii\compilepart@iiitrue
% \newif\ifcompilepart@iv\compilepart@ivfalse

%% Table of contents depth control
\setcounter{tocdepth}{3}

%% Cross-reference formatting
\providecommand{\soiArtRef}[1]{Article~\ref{art:#1}}
\providecommand{\soiPartRef}[1]{Part~\ref{part:#1}}
\providecommand{\soiSchedRef}[1]{Schedule~\ref{sched:#1}}

%% Special constitutional formatting
\providecommand{\soiPreambleFormat}{\centering\large\itshape}
\providecommand{\soiSignatureFormat}{\raggedleft\normalsize}

%% End of configuration}{%
            \IfFileExists{../../config.tex}{%% Configuration file for Source of India LaTeX class
%% This file contains customizable settings and preferences

%% Amendment display preferences
% Uncomment the following to hide amendments globally
% \@soishowamendmentsfalse

%% Typography fine-tuning
\setlength{\parskip}{\baselineskip}
\setlength{\parindent}{0pt}

%% Footnote spacing adjustments
\setlength{\footnotesep}{0.75em}
\addtolength{\skip\footins}{0.25em}

%% Enhanced spacing for constitutional elements
\newlength{\articlespacing}
\setlength{\articlespacing}{1.5\baselineskip}

\newlength{\clausespacing}
\setlength{\clausespacing}{0.75\baselineskip}

%% Custom formatting for special constitutional terms
\providecommand{\soiConstitutionTerm}[1]{\textsc{#1}}
\providecommand{\soiActCitation}[1]{\textit{#1}}

%% Header customization
\providecommand{\soiHeaderFont}{\small\itshape}

%% Page layout fine adjustments
\addtolength{\textheight}{0.5cm}
\addtolength{\textwidth}{0.25cm}

%% Constitutional document metadata
\providecommand{\soiConstitutionTitle}{THE CONSTITUTION OF INDIA}
\providecommand{\soiConstitutionSubtitle}{As amended up to the One Hundred and Fifth Amendment Act, 2021}
\providecommand{\soiConstitutionDate}{26th November, 1949}

%% Amendment citation formatting
\providecommand{\soiAmendmentCite}[2]{%
    Constitution (#1 Amendment) Act, #2%
}

%% Effective date formatting
\providecommand{\soiEffectiveDate}[1]{%
    \textit{w.e.f.} #1%
}

%% Development mode settings
\if@soidraftmode
    % Show additional information in draft mode
    \fancyfoot[LE]{\tiny Draft Mode}
    \fancyfoot[RO]{\tiny \today}
\fi

%% Conditional compilation settings
% Uncomment to compile only specific parts
% \newif\ifcompilepart@i\compilepart@itrue
% \newif\ifcompilepart@iii\compilepart@iiitrue
% \newif\ifcompilepart@iv\compilepart@ivfalse

%% Table of contents depth control
\setcounter{tocdepth}{3}

%% Cross-reference formatting
\providecommand{\soiArtRef}[1]{Article~\ref{art:#1}}
\providecommand{\soiPartRef}[1]{Part~\ref{part:#1}}
\providecommand{\soiSchedRef}[1]{Schedule~\ref{sched:#1}}

%% Special constitutional formatting
\providecommand{\soiPreambleFormat}{\centering\large\itshape}
\providecommand{\soiSignatureFormat}{\raggedleft\normalsize}

%% End of configuration}{%
                % Use default config if none found
            }%
        }%
    }
    
    \begin{document}
    \def\soistandalone{true}
\fi

\soiArticle{Name and territory of the Union}{%
    \soiClause{India, that is Bharat, shall be a Union of States.}
    
    \soiClause{The States and the territories thereof shall be as specified in the \soiAmendment{First Schedule}\soiAmendmentData{Constitution (Seventh Amendment) Act, 1956, s. 2 \soiWef{1-11-1956}}.}
    
    \soiClause{The territory of India shall comprise—
        \soiSubClause{the territories of the States;}
        \soiSubClause{the Union territories specified in the First Schedule; and}
        \soiSubClause{such other territories as may be acquired.}
    }
}

\ifdefined\soistandalone\end{document}\fi}{%
        \IfFileExists{content/articles/article_001.tex}{%% Article 1: Name and territory of the Union
%% Can be compiled independently from any directory level

% Check if this is being compiled as a standalone document
\ifx\documentclass\undefined\else
    % Standalone compilation
    \documentclass[a4paper,showamendments]{soi}
    
    % Path-aware config loading
    \IfFileExists{config.tex}{%% Configuration file for Source of India LaTeX class
%% This file contains customizable settings and preferences

%% Amendment display preferences
% Uncomment the following to hide amendments globally
% \@soishowamendmentsfalse

%% Typography fine-tuning
\setlength{\parskip}{\baselineskip}
\setlength{\parindent}{0pt}

%% Footnote spacing adjustments
\setlength{\footnotesep}{0.75em}
\addtolength{\skip\footins}{0.25em}

%% Enhanced spacing for constitutional elements
\newlength{\articlespacing}
\setlength{\articlespacing}{1.5\baselineskip}

\newlength{\clausespacing}
\setlength{\clausespacing}{0.75\baselineskip}

%% Custom formatting for special constitutional terms
\providecommand{\soiConstitutionTerm}[1]{\textsc{#1}}
\providecommand{\soiActCitation}[1]{\textit{#1}}

%% Header customization
\providecommand{\soiHeaderFont}{\small\itshape}

%% Page layout fine adjustments
\addtolength{\textheight}{0.5cm}
\addtolength{\textwidth}{0.25cm}

%% Constitutional document metadata
\providecommand{\soiConstitutionTitle}{THE CONSTITUTION OF INDIA}
\providecommand{\soiConstitutionSubtitle}{As amended up to the One Hundred and Fifth Amendment Act, 2021}
\providecommand{\soiConstitutionDate}{26th November, 1949}

%% Amendment citation formatting
\providecommand{\soiAmendmentCite}[2]{%
    Constitution (#1 Amendment) Act, #2%
}

%% Effective date formatting
\providecommand{\soiEffectiveDate}[1]{%
    \textit{w.e.f.} #1%
}

%% Development mode settings
\if@soidraftmode
    % Show additional information in draft mode
    \fancyfoot[LE]{\tiny Draft Mode}
    \fancyfoot[RO]{\tiny \today}
\fi

%% Conditional compilation settings
% Uncomment to compile only specific parts
% \newif\ifcompilepart@i\compilepart@itrue
% \newif\ifcompilepart@iii\compilepart@iiitrue
% \newif\ifcompilepart@iv\compilepart@ivfalse

%% Table of contents depth control
\setcounter{tocdepth}{3}

%% Cross-reference formatting
\providecommand{\soiArtRef}[1]{Article~\ref{art:#1}}
\providecommand{\soiPartRef}[1]{Part~\ref{part:#1}}
\providecommand{\soiSchedRef}[1]{Schedule~\ref{sched:#1}}

%% Special constitutional formatting
\providecommand{\soiPreambleFormat}{\centering\large\itshape}
\providecommand{\soiSignatureFormat}{\raggedleft\normalsize}

%% End of configuration}{%
        \IfFileExists{../config.tex}{%% Configuration file for Source of India LaTeX class
%% This file contains customizable settings and preferences

%% Amendment display preferences
% Uncomment the following to hide amendments globally
% \@soishowamendmentsfalse

%% Typography fine-tuning
\setlength{\parskip}{\baselineskip}
\setlength{\parindent}{0pt}

%% Footnote spacing adjustments
\setlength{\footnotesep}{0.75em}
\addtolength{\skip\footins}{0.25em}

%% Enhanced spacing for constitutional elements
\newlength{\articlespacing}
\setlength{\articlespacing}{1.5\baselineskip}

\newlength{\clausespacing}
\setlength{\clausespacing}{0.75\baselineskip}

%% Custom formatting for special constitutional terms
\providecommand{\soiConstitutionTerm}[1]{\textsc{#1}}
\providecommand{\soiActCitation}[1]{\textit{#1}}

%% Header customization
\providecommand{\soiHeaderFont}{\small\itshape}

%% Page layout fine adjustments
\addtolength{\textheight}{0.5cm}
\addtolength{\textwidth}{0.25cm}

%% Constitutional document metadata
\providecommand{\soiConstitutionTitle}{THE CONSTITUTION OF INDIA}
\providecommand{\soiConstitutionSubtitle}{As amended up to the One Hundred and Fifth Amendment Act, 2021}
\providecommand{\soiConstitutionDate}{26th November, 1949}

%% Amendment citation formatting
\providecommand{\soiAmendmentCite}[2]{%
    Constitution (#1 Amendment) Act, #2%
}

%% Effective date formatting
\providecommand{\soiEffectiveDate}[1]{%
    \textit{w.e.f.} #1%
}

%% Development mode settings
\if@soidraftmode
    % Show additional information in draft mode
    \fancyfoot[LE]{\tiny Draft Mode}
    \fancyfoot[RO]{\tiny \today}
\fi

%% Conditional compilation settings
% Uncomment to compile only specific parts
% \newif\ifcompilepart@i\compilepart@itrue
% \newif\ifcompilepart@iii\compilepart@iiitrue
% \newif\ifcompilepart@iv\compilepart@ivfalse

%% Table of contents depth control
\setcounter{tocdepth}{3}

%% Cross-reference formatting
\providecommand{\soiArtRef}[1]{Article~\ref{art:#1}}
\providecommand{\soiPartRef}[1]{Part~\ref{part:#1}}
\providecommand{\soiSchedRef}[1]{Schedule~\ref{sched:#1}}

%% Special constitutional formatting
\providecommand{\soiPreambleFormat}{\centering\large\itshape}
\providecommand{\soiSignatureFormat}{\raggedleft\normalsize}

%% End of configuration}{%
            \IfFileExists{../../config.tex}{%% Configuration file for Source of India LaTeX class
%% This file contains customizable settings and preferences

%% Amendment display preferences
% Uncomment the following to hide amendments globally
% \@soishowamendmentsfalse

%% Typography fine-tuning
\setlength{\parskip}{\baselineskip}
\setlength{\parindent}{0pt}

%% Footnote spacing adjustments
\setlength{\footnotesep}{0.75em}
\addtolength{\skip\footins}{0.25em}

%% Enhanced spacing for constitutional elements
\newlength{\articlespacing}
\setlength{\articlespacing}{1.5\baselineskip}

\newlength{\clausespacing}
\setlength{\clausespacing}{0.75\baselineskip}

%% Custom formatting for special constitutional terms
\providecommand{\soiConstitutionTerm}[1]{\textsc{#1}}
\providecommand{\soiActCitation}[1]{\textit{#1}}

%% Header customization
\providecommand{\soiHeaderFont}{\small\itshape}

%% Page layout fine adjustments
\addtolength{\textheight}{0.5cm}
\addtolength{\textwidth}{0.25cm}

%% Constitutional document metadata
\providecommand{\soiConstitutionTitle}{THE CONSTITUTION OF INDIA}
\providecommand{\soiConstitutionSubtitle}{As amended up to the One Hundred and Fifth Amendment Act, 2021}
\providecommand{\soiConstitutionDate}{26th November, 1949}

%% Amendment citation formatting
\providecommand{\soiAmendmentCite}[2]{%
    Constitution (#1 Amendment) Act, #2%
}

%% Effective date formatting
\providecommand{\soiEffectiveDate}[1]{%
    \textit{w.e.f.} #1%
}

%% Development mode settings
\if@soidraftmode
    % Show additional information in draft mode
    \fancyfoot[LE]{\tiny Draft Mode}
    \fancyfoot[RO]{\tiny \today}
\fi

%% Conditional compilation settings
% Uncomment to compile only specific parts
% \newif\ifcompilepart@i\compilepart@itrue
% \newif\ifcompilepart@iii\compilepart@iiitrue
% \newif\ifcompilepart@iv\compilepart@ivfalse

%% Table of contents depth control
\setcounter{tocdepth}{3}

%% Cross-reference formatting
\providecommand{\soiArtRef}[1]{Article~\ref{art:#1}}
\providecommand{\soiPartRef}[1]{Part~\ref{part:#1}}
\providecommand{\soiSchedRef}[1]{Schedule~\ref{sched:#1}}

%% Special constitutional formatting
\providecommand{\soiPreambleFormat}{\centering\large\itshape}
\providecommand{\soiSignatureFormat}{\raggedleft\normalsize}

%% End of configuration}{%
                % Use default config if none found
            }%
        }%
    }
    
    \begin{document}
    \def\soistandalone{true}
\fi

\soiArticle{Name and territory of the Union}{%
    \soiClause{India, that is Bharat, shall be a Union of States.}
    
    \soiClause{The States and the territories thereof shall be as specified in the \soiAmendment{First Schedule}\soiAmendmentData{Constitution (Seventh Amendment) Act, 1956, s. 2 \soiWef{1-11-1956}}.}
    
    \soiClause{The territory of India shall comprise—
        \soiSubClause{the territories of the States;}
        \soiSubClause{the Union territories specified in the First Schedule; and}
        \soiSubClause{such other territories as may be acquired.}
    }
}

\ifdefined\soistandalone\end{document}\fi}{%
            \ClassError{soi}{Cannot find article_001.tex}{Check the file path}%
        }%
    }%
}

% Note: Additional articles 2, 3, 4 would be included here
% Similar pattern for other articles

\ifdefined\soistandalone\end{document}\fi

% Part III: Fundamental Rights (selected articles)
%% Part III: Fundamental Rights

\soiPart{III}{FUNDAMENTAL RIGHTS}

% Chapter: Right to Equality
\soiChapter{Right to Equality}
%% Article 14: Equality before law
%% Can be compiled independently from any directory level

% Path-aware configuration loading
\@ifundefined{documentclass}{
    \documentclass[a4paper,showamendments]{soi}
    \IfFileExists{config.tex}{%% Configuration file for Source of India LaTeX class
%% This file contains customizable settings and preferences

%% Amendment display preferences
% Uncomment the following to hide amendments globally
% \@soishowamendmentsfalse

%% Typography fine-tuning
\setlength{\parskip}{\baselineskip}
\setlength{\parindent}{0pt}

%% Footnote spacing adjustments
\setlength{\footnotesep}{0.75em}
\addtolength{\skip\footins}{0.25em}

%% Enhanced spacing for constitutional elements
\newlength{\articlespacing}
\setlength{\articlespacing}{1.5\baselineskip}

\newlength{\clausespacing}
\setlength{\clausespacing}{0.75\baselineskip}

%% Custom formatting for special constitutional terms
\providecommand{\soiConstitutionTerm}[1]{\textsc{#1}}
\providecommand{\soiActCitation}[1]{\textit{#1}}

%% Header customization
\providecommand{\soiHeaderFont}{\small\itshape}

%% Page layout fine adjustments
\addtolength{\textheight}{0.5cm}
\addtolength{\textwidth}{0.25cm}

%% Constitutional document metadata
\providecommand{\soiConstitutionTitle}{THE CONSTITUTION OF INDIA}
\providecommand{\soiConstitutionSubtitle}{As amended up to the One Hundred and Fifth Amendment Act, 2021}
\providecommand{\soiConstitutionDate}{26th November, 1949}

%% Amendment citation formatting
\providecommand{\soiAmendmentCite}[2]{%
    Constitution (#1 Amendment) Act, #2%
}

%% Effective date formatting
\providecommand{\soiEffectiveDate}[1]{%
    \textit{w.e.f.} #1%
}

%% Development mode settings
\if@soidraftmode
    % Show additional information in draft mode
    \fancyfoot[LE]{\tiny Draft Mode}
    \fancyfoot[RO]{\tiny \today}
\fi

%% Conditional compilation settings
% Uncomment to compile only specific parts
% \newif\ifcompilepart@i\compilepart@itrue
% \newif\ifcompilepart@iii\compilepart@iiitrue
% \newif\ifcompilepart@iv\compilepart@ivfalse

%% Table of contents depth control
\setcounter{tocdepth}{3}

%% Cross-reference formatting
\providecommand{\soiArtRef}[1]{Article~\ref{art:#1}}
\providecommand{\soiPartRef}[1]{Part~\ref{part:#1}}
\providecommand{\soiSchedRef}[1]{Schedule~\ref{sched:#1}}

%% Special constitutional formatting
\providecommand{\soiPreambleFormat}{\centering\large\itshape}
\providecommand{\soiSignatureFormat}{\raggedleft\normalsize}

%% End of configuration}{\IfFileExists{../config.tex}{%% Configuration file for Source of India LaTeX class
%% This file contains customizable settings and preferences

%% Amendment display preferences
% Uncomment the following to hide amendments globally
% \@soishowamendmentsfalse

%% Typography fine-tuning
\setlength{\parskip}{\baselineskip}
\setlength{\parindent}{0pt}

%% Footnote spacing adjustments
\setlength{\footnotesep}{0.75em}
\addtolength{\skip\footins}{0.25em}

%% Enhanced spacing for constitutional elements
\newlength{\articlespacing}
\setlength{\articlespacing}{1.5\baselineskip}

\newlength{\clausespacing}
\setlength{\clausespacing}{0.75\baselineskip}

%% Custom formatting for special constitutional terms
\providecommand{\soiConstitutionTerm}[1]{\textsc{#1}}
\providecommand{\soiActCitation}[1]{\textit{#1}}

%% Header customization
\providecommand{\soiHeaderFont}{\small\itshape}

%% Page layout fine adjustments
\addtolength{\textheight}{0.5cm}
\addtolength{\textwidth}{0.25cm}

%% Constitutional document metadata
\providecommand{\soiConstitutionTitle}{THE CONSTITUTION OF INDIA}
\providecommand{\soiConstitutionSubtitle}{As amended up to the One Hundred and Fifth Amendment Act, 2021}
\providecommand{\soiConstitutionDate}{26th November, 1949}

%% Amendment citation formatting
\providecommand{\soiAmendmentCite}[2]{%
    Constitution (#1 Amendment) Act, #2%
}

%% Effective date formatting
\providecommand{\soiEffectiveDate}[1]{%
    \textit{w.e.f.} #1%
}

%% Development mode settings
\if@soidraftmode
    % Show additional information in draft mode
    \fancyfoot[LE]{\tiny Draft Mode}
    \fancyfoot[RO]{\tiny \today}
\fi

%% Conditional compilation settings
% Uncomment to compile only specific parts
% \newif\ifcompilepart@i\compilepart@itrue
% \newif\ifcompilepart@iii\compilepart@iiitrue
% \newif\ifcompilepart@iv\compilepart@ivfalse

%% Table of contents depth control
\setcounter{tocdepth}{3}

%% Cross-reference formatting
\providecommand{\soiArtRef}[1]{Article~\ref{art:#1}}
\providecommand{\soiPartRef}[1]{Part~\ref{part:#1}}
\providecommand{\soiSchedRef}[1]{Schedule~\ref{sched:#1}}

%% Special constitutional formatting
\providecommand{\soiPreambleFormat}{\centering\large\itshape}
\providecommand{\soiSignatureFormat}{\raggedleft\normalsize}

%% End of configuration}{\IfFileExists{../../config.tex}{%% Configuration file for Source of India LaTeX class
%% This file contains customizable settings and preferences

%% Amendment display preferences
% Uncomment the following to hide amendments globally
% \@soishowamendmentsfalse

%% Typography fine-tuning
\setlength{\parskip}{\baselineskip}
\setlength{\parindent}{0pt}

%% Footnote spacing adjustments
\setlength{\footnotesep}{0.75em}
\addtolength{\skip\footins}{0.25em}

%% Enhanced spacing for constitutional elements
\newlength{\articlespacing}
\setlength{\articlespacing}{1.5\baselineskip}

\newlength{\clausespacing}
\setlength{\clausespacing}{0.75\baselineskip}

%% Custom formatting for special constitutional terms
\providecommand{\soiConstitutionTerm}[1]{\textsc{#1}}
\providecommand{\soiActCitation}[1]{\textit{#1}}

%% Header customization
\providecommand{\soiHeaderFont}{\small\itshape}

%% Page layout fine adjustments
\addtolength{\textheight}{0.5cm}
\addtolength{\textwidth}{0.25cm}

%% Constitutional document metadata
\providecommand{\soiConstitutionTitle}{THE CONSTITUTION OF INDIA}
\providecommand{\soiConstitutionSubtitle}{As amended up to the One Hundred and Fifth Amendment Act, 2021}
\providecommand{\soiConstitutionDate}{26th November, 1949}

%% Amendment citation formatting
\providecommand{\soiAmendmentCite}[2]{%
    Constitution (#1 Amendment) Act, #2%
}

%% Effective date formatting
\providecommand{\soiEffectiveDate}[1]{%
    \textit{w.e.f.} #1%
}

%% Development mode settings
\if@soidraftmode
    % Show additional information in draft mode
    \fancyfoot[LE]{\tiny Draft Mode}
    \fancyfoot[RO]{\tiny \today}
\fi

%% Conditional compilation settings
% Uncomment to compile only specific parts
% \newif\ifcompilepart@i\compilepart@itrue
% \newif\ifcompilepart@iii\compilepart@iiitrue
% \newif\ifcompilepart@iv\compilepart@ivfalse

%% Table of contents depth control
\setcounter{tocdepth}{3}

%% Cross-reference formatting
\providecommand{\soiArtRef}[1]{Article~\ref{art:#1}}
\providecommand{\soiPartRef}[1]{Part~\ref{part:#1}}
\providecommand{\soiSchedRef}[1]{Schedule~\ref{sched:#1}}

%% Special constitutional formatting
\providecommand{\soiPreambleFormat}{\centering\large\itshape}
\providecommand{\soiSignatureFormat}{\raggedleft\normalsize}

%% End of configuration}{}}}
    \begin{document}
}{}

\soiArticle{Equality before law}{%
    The State shall not deny to any person equality before the law or the equal protection of the laws within the territory of India.
}

\@ifundefined{documentclass}{}{\end{document}}
%% Article 15: Prohibition of discrimination on grounds of religion, race, caste, sex or place of birth
%% Can be compiled independently from any directory level

% Path-aware configuration loading
\@ifundefined{documentclass}{
    \documentclass[a4paper,showamendments]{soi}
    \IfFileExists{config.tex}{%% Configuration file for Source of India LaTeX class
%% This file contains customizable settings and preferences

%% Amendment display preferences
% Uncomment the following to hide amendments globally
% \@soishowamendmentsfalse

%% Typography fine-tuning
\setlength{\parskip}{\baselineskip}
\setlength{\parindent}{0pt}

%% Footnote spacing adjustments
\setlength{\footnotesep}{0.75em}
\addtolength{\skip\footins}{0.25em}

%% Enhanced spacing for constitutional elements
\newlength{\articlespacing}
\setlength{\articlespacing}{1.5\baselineskip}

\newlength{\clausespacing}
\setlength{\clausespacing}{0.75\baselineskip}

%% Custom formatting for special constitutional terms
\providecommand{\soiConstitutionTerm}[1]{\textsc{#1}}
\providecommand{\soiActCitation}[1]{\textit{#1}}

%% Header customization
\providecommand{\soiHeaderFont}{\small\itshape}

%% Page layout fine adjustments
\addtolength{\textheight}{0.5cm}
\addtolength{\textwidth}{0.25cm}

%% Constitutional document metadata
\providecommand{\soiConstitutionTitle}{THE CONSTITUTION OF INDIA}
\providecommand{\soiConstitutionSubtitle}{As amended up to the One Hundred and Fifth Amendment Act, 2021}
\providecommand{\soiConstitutionDate}{26th November, 1949}

%% Amendment citation formatting
\providecommand{\soiAmendmentCite}[2]{%
    Constitution (#1 Amendment) Act, #2%
}

%% Effective date formatting
\providecommand{\soiEffectiveDate}[1]{%
    \textit{w.e.f.} #1%
}

%% Development mode settings
\if@soidraftmode
    % Show additional information in draft mode
    \fancyfoot[LE]{\tiny Draft Mode}
    \fancyfoot[RO]{\tiny \today}
\fi

%% Conditional compilation settings
% Uncomment to compile only specific parts
% \newif\ifcompilepart@i\compilepart@itrue
% \newif\ifcompilepart@iii\compilepart@iiitrue
% \newif\ifcompilepart@iv\compilepart@ivfalse

%% Table of contents depth control
\setcounter{tocdepth}{3}

%% Cross-reference formatting
\providecommand{\soiArtRef}[1]{Article~\ref{art:#1}}
\providecommand{\soiPartRef}[1]{Part~\ref{part:#1}}
\providecommand{\soiSchedRef}[1]{Schedule~\ref{sched:#1}}

%% Special constitutional formatting
\providecommand{\soiPreambleFormat}{\centering\large\itshape}
\providecommand{\soiSignatureFormat}{\raggedleft\normalsize}

%% End of configuration}{\IfFileExists{../config.tex}{%% Configuration file for Source of India LaTeX class
%% This file contains customizable settings and preferences

%% Amendment display preferences
% Uncomment the following to hide amendments globally
% \@soishowamendmentsfalse

%% Typography fine-tuning
\setlength{\parskip}{\baselineskip}
\setlength{\parindent}{0pt}

%% Footnote spacing adjustments
\setlength{\footnotesep}{0.75em}
\addtolength{\skip\footins}{0.25em}

%% Enhanced spacing for constitutional elements
\newlength{\articlespacing}
\setlength{\articlespacing}{1.5\baselineskip}

\newlength{\clausespacing}
\setlength{\clausespacing}{0.75\baselineskip}

%% Custom formatting for special constitutional terms
\providecommand{\soiConstitutionTerm}[1]{\textsc{#1}}
\providecommand{\soiActCitation}[1]{\textit{#1}}

%% Header customization
\providecommand{\soiHeaderFont}{\small\itshape}

%% Page layout fine adjustments
\addtolength{\textheight}{0.5cm}
\addtolength{\textwidth}{0.25cm}

%% Constitutional document metadata
\providecommand{\soiConstitutionTitle}{THE CONSTITUTION OF INDIA}
\providecommand{\soiConstitutionSubtitle}{As amended up to the One Hundred and Fifth Amendment Act, 2021}
\providecommand{\soiConstitutionDate}{26th November, 1949}

%% Amendment citation formatting
\providecommand{\soiAmendmentCite}[2]{%
    Constitution (#1 Amendment) Act, #2%
}

%% Effective date formatting
\providecommand{\soiEffectiveDate}[1]{%
    \textit{w.e.f.} #1%
}

%% Development mode settings
\if@soidraftmode
    % Show additional information in draft mode
    \fancyfoot[LE]{\tiny Draft Mode}
    \fancyfoot[RO]{\tiny \today}
\fi

%% Conditional compilation settings
% Uncomment to compile only specific parts
% \newif\ifcompilepart@i\compilepart@itrue
% \newif\ifcompilepart@iii\compilepart@iiitrue
% \newif\ifcompilepart@iv\compilepart@ivfalse

%% Table of contents depth control
\setcounter{tocdepth}{3}

%% Cross-reference formatting
\providecommand{\soiArtRef}[1]{Article~\ref{art:#1}}
\providecommand{\soiPartRef}[1]{Part~\ref{part:#1}}
\providecommand{\soiSchedRef}[1]{Schedule~\ref{sched:#1}}

%% Special constitutional formatting
\providecommand{\soiPreambleFormat}{\centering\large\itshape}
\providecommand{\soiSignatureFormat}{\raggedleft\normalsize}

%% End of configuration}{\IfFileExists{../../config.tex}{%% Configuration file for Source of India LaTeX class
%% This file contains customizable settings and preferences

%% Amendment display preferences
% Uncomment the following to hide amendments globally
% \@soishowamendmentsfalse

%% Typography fine-tuning
\setlength{\parskip}{\baselineskip}
\setlength{\parindent}{0pt}

%% Footnote spacing adjustments
\setlength{\footnotesep}{0.75em}
\addtolength{\skip\footins}{0.25em}

%% Enhanced spacing for constitutional elements
\newlength{\articlespacing}
\setlength{\articlespacing}{1.5\baselineskip}

\newlength{\clausespacing}
\setlength{\clausespacing}{0.75\baselineskip}

%% Custom formatting for special constitutional terms
\providecommand{\soiConstitutionTerm}[1]{\textsc{#1}}
\providecommand{\soiActCitation}[1]{\textit{#1}}

%% Header customization
\providecommand{\soiHeaderFont}{\small\itshape}

%% Page layout fine adjustments
\addtolength{\textheight}{0.5cm}
\addtolength{\textwidth}{0.25cm}

%% Constitutional document metadata
\providecommand{\soiConstitutionTitle}{THE CONSTITUTION OF INDIA}
\providecommand{\soiConstitutionSubtitle}{As amended up to the One Hundred and Fifth Amendment Act, 2021}
\providecommand{\soiConstitutionDate}{26th November, 1949}

%% Amendment citation formatting
\providecommand{\soiAmendmentCite}[2]{%
    Constitution (#1 Amendment) Act, #2%
}

%% Effective date formatting
\providecommand{\soiEffectiveDate}[1]{%
    \textit{w.e.f.} #1%
}

%% Development mode settings
\if@soidraftmode
    % Show additional information in draft mode
    \fancyfoot[LE]{\tiny Draft Mode}
    \fancyfoot[RO]{\tiny \today}
\fi

%% Conditional compilation settings
% Uncomment to compile only specific parts
% \newif\ifcompilepart@i\compilepart@itrue
% \newif\ifcompilepart@iii\compilepart@iiitrue
% \newif\ifcompilepart@iv\compilepart@ivfalse

%% Table of contents depth control
\setcounter{tocdepth}{3}

%% Cross-reference formatting
\providecommand{\soiArtRef}[1]{Article~\ref{art:#1}}
\providecommand{\soiPartRef}[1]{Part~\ref{part:#1}}
\providecommand{\soiSchedRef}[1]{Schedule~\ref{sched:#1}}

%% Special constitutional formatting
\providecommand{\soiPreambleFormat}{\centering\large\itshape}
\providecommand{\soiSignatureFormat}{\raggedleft\normalsize}

%% End of configuration}{}}}
    \begin{document}
}{}

\soiArticle{Prohibition of discrimination on grounds of religion, race, caste, sex or place of birth}{%
    \soiClause{The State shall not discriminate against any citizen on grounds only of religion, race, caste, sex, place of birth or any of them.}
    
    \soiClause{No citizen shall, on grounds only of religion, race, caste, sex, place of birth or any of them, be subject to any disability, liability, restriction or condition with regard to—
        \soiSubClause{access to shops, public restaurants, hotels and places of public entertainment; or}
        \soiSubClause{the use of wells, tanks, bathing ghats, roads and places of public resort maintained wholly or partly out of State funds or dedicated to the use of the general public.}
    }
    
    \soiClause{Nothing in this article shall prevent the State from making any special provision for women and children.}
    
    \soiClause{\soiAmendment{Nothing in this article or in clause (2) of article 29 shall prevent the State from making any special provision for the advancement of any socially and educationally backward classes of citizens or for the Scheduled Castes and the Scheduled Tribes.}\soiAmendmentData{Constitution (First Amendment) Act, 1951, s. 2 \soiWef{18-6-1951}}}
    
    \soiClause{\soiAmendment{Nothing in this article or in sub-clause (g) of clause (1) of article 19 shall prevent the State from making any special provision, by law, for the advancement of any socially and educationally backward classes of citizens or for the Scheduled Castes or the Scheduled Tribes in so far as such special provisions relate to their admission to educational institutions including private educational institutions, whether aided or unaided by the State, other than the minority educational institutions referred to in clause (1) of article 30.}\soiAmendmentData{Constitution (Ninety-third Amendment) Act, 2005, s. 2 \soiWef{20-1-2006}}}
    
    \soiClause{\soiAmendment{Nothing in this article or in clause (2) of article 29 or in clause (3) of article 30 shall prevent the State from making any provision for the reservation of appointments or posts in favour of any backward class of citizens which, in the opinion of the State, is not adequately represented in the services under the State.}\soiAmendmentData{Constitution (One Hundred and Third Amendment) Act, 2019, s. 2 \soiWef{14-1-2019}}}
}

\@ifundefined{documentclass}{}{\end{document}}
% Note: Articles 16, 17, 18 would also be included here

% Chapter: Right to Freedom  
% \soiChapter{Right to Freedom}
% \input{content/articles/article_019}
% \input{content/articles/article_020}
%% Article 21A: Right to education
%% Can be compiled independently from any directory level

% Path-aware configuration loading
\@ifundefined{documentclass}{
    \documentclass[a4paper,showamendments]{soi}
    \IfFileExists{config.tex}{%% Configuration file for Source of India LaTeX class
%% This file contains customizable settings and preferences

%% Amendment display preferences
% Uncomment the following to hide amendments globally
% \@soishowamendmentsfalse

%% Typography fine-tuning
\setlength{\parskip}{\baselineskip}
\setlength{\parindent}{0pt}

%% Footnote spacing adjustments
\setlength{\footnotesep}{0.75em}
\addtolength{\skip\footins}{0.25em}

%% Enhanced spacing for constitutional elements
\newlength{\articlespacing}
\setlength{\articlespacing}{1.5\baselineskip}

\newlength{\clausespacing}
\setlength{\clausespacing}{0.75\baselineskip}

%% Custom formatting for special constitutional terms
\providecommand{\soiConstitutionTerm}[1]{\textsc{#1}}
\providecommand{\soiActCitation}[1]{\textit{#1}}

%% Header customization
\providecommand{\soiHeaderFont}{\small\itshape}

%% Page layout fine adjustments
\addtolength{\textheight}{0.5cm}
\addtolength{\textwidth}{0.25cm}

%% Constitutional document metadata
\providecommand{\soiConstitutionTitle}{THE CONSTITUTION OF INDIA}
\providecommand{\soiConstitutionSubtitle}{As amended up to the One Hundred and Fifth Amendment Act, 2021}
\providecommand{\soiConstitutionDate}{26th November, 1949}

%% Amendment citation formatting
\providecommand{\soiAmendmentCite}[2]{%
    Constitution (#1 Amendment) Act, #2%
}

%% Effective date formatting
\providecommand{\soiEffectiveDate}[1]{%
    \textit{w.e.f.} #1%
}

%% Development mode settings
\if@soidraftmode
    % Show additional information in draft mode
    \fancyfoot[LE]{\tiny Draft Mode}
    \fancyfoot[RO]{\tiny \today}
\fi

%% Conditional compilation settings
% Uncomment to compile only specific parts
% \newif\ifcompilepart@i\compilepart@itrue
% \newif\ifcompilepart@iii\compilepart@iiitrue
% \newif\ifcompilepart@iv\compilepart@ivfalse

%% Table of contents depth control
\setcounter{tocdepth}{3}

%% Cross-reference formatting
\providecommand{\soiArtRef}[1]{Article~\ref{art:#1}}
\providecommand{\soiPartRef}[1]{Part~\ref{part:#1}}
\providecommand{\soiSchedRef}[1]{Schedule~\ref{sched:#1}}

%% Special constitutional formatting
\providecommand{\soiPreambleFormat}{\centering\large\itshape}
\providecommand{\soiSignatureFormat}{\raggedleft\normalsize}

%% End of configuration}{\IfFileExists{../config.tex}{%% Configuration file for Source of India LaTeX class
%% This file contains customizable settings and preferences

%% Amendment display preferences
% Uncomment the following to hide amendments globally
% \@soishowamendmentsfalse

%% Typography fine-tuning
\setlength{\parskip}{\baselineskip}
\setlength{\parindent}{0pt}

%% Footnote spacing adjustments
\setlength{\footnotesep}{0.75em}
\addtolength{\skip\footins}{0.25em}

%% Enhanced spacing for constitutional elements
\newlength{\articlespacing}
\setlength{\articlespacing}{1.5\baselineskip}

\newlength{\clausespacing}
\setlength{\clausespacing}{0.75\baselineskip}

%% Custom formatting for special constitutional terms
\providecommand{\soiConstitutionTerm}[1]{\textsc{#1}}
\providecommand{\soiActCitation}[1]{\textit{#1}}

%% Header customization
\providecommand{\soiHeaderFont}{\small\itshape}

%% Page layout fine adjustments
\addtolength{\textheight}{0.5cm}
\addtolength{\textwidth}{0.25cm}

%% Constitutional document metadata
\providecommand{\soiConstitutionTitle}{THE CONSTITUTION OF INDIA}
\providecommand{\soiConstitutionSubtitle}{As amended up to the One Hundred and Fifth Amendment Act, 2021}
\providecommand{\soiConstitutionDate}{26th November, 1949}

%% Amendment citation formatting
\providecommand{\soiAmendmentCite}[2]{%
    Constitution (#1 Amendment) Act, #2%
}

%% Effective date formatting
\providecommand{\soiEffectiveDate}[1]{%
    \textit{w.e.f.} #1%
}

%% Development mode settings
\if@soidraftmode
    % Show additional information in draft mode
    \fancyfoot[LE]{\tiny Draft Mode}
    \fancyfoot[RO]{\tiny \today}
\fi

%% Conditional compilation settings
% Uncomment to compile only specific parts
% \newif\ifcompilepart@i\compilepart@itrue
% \newif\ifcompilepart@iii\compilepart@iiitrue
% \newif\ifcompilepart@iv\compilepart@ivfalse

%% Table of contents depth control
\setcounter{tocdepth}{3}

%% Cross-reference formatting
\providecommand{\soiArtRef}[1]{Article~\ref{art:#1}}
\providecommand{\soiPartRef}[1]{Part~\ref{part:#1}}
\providecommand{\soiSchedRef}[1]{Schedule~\ref{sched:#1}}

%% Special constitutional formatting
\providecommand{\soiPreambleFormat}{\centering\large\itshape}
\providecommand{\soiSignatureFormat}{\raggedleft\normalsize}

%% End of configuration}{\IfFileExists{../../config.tex}{%% Configuration file for Source of India LaTeX class
%% This file contains customizable settings and preferences

%% Amendment display preferences
% Uncomment the following to hide amendments globally
% \@soishowamendmentsfalse

%% Typography fine-tuning
\setlength{\parskip}{\baselineskip}
\setlength{\parindent}{0pt}

%% Footnote spacing adjustments
\setlength{\footnotesep}{0.75em}
\addtolength{\skip\footins}{0.25em}

%% Enhanced spacing for constitutional elements
\newlength{\articlespacing}
\setlength{\articlespacing}{1.5\baselineskip}

\newlength{\clausespacing}
\setlength{\clausespacing}{0.75\baselineskip}

%% Custom formatting for special constitutional terms
\providecommand{\soiConstitutionTerm}[1]{\textsc{#1}}
\providecommand{\soiActCitation}[1]{\textit{#1}}

%% Header customization
\providecommand{\soiHeaderFont}{\small\itshape}

%% Page layout fine adjustments
\addtolength{\textheight}{0.5cm}
\addtolength{\textwidth}{0.25cm}

%% Constitutional document metadata
\providecommand{\soiConstitutionTitle}{THE CONSTITUTION OF INDIA}
\providecommand{\soiConstitutionSubtitle}{As amended up to the One Hundred and Fifth Amendment Act, 2021}
\providecommand{\soiConstitutionDate}{26th November, 1949}

%% Amendment citation formatting
\providecommand{\soiAmendmentCite}[2]{%
    Constitution (#1 Amendment) Act, #2%
}

%% Effective date formatting
\providecommand{\soiEffectiveDate}[1]{%
    \textit{w.e.f.} #1%
}

%% Development mode settings
\if@soidraftmode
    % Show additional information in draft mode
    \fancyfoot[LE]{\tiny Draft Mode}
    \fancyfoot[RO]{\tiny \today}
\fi

%% Conditional compilation settings
% Uncomment to compile only specific parts
% \newif\ifcompilepart@i\compilepart@itrue
% \newif\ifcompilepart@iii\compilepart@iiitrue
% \newif\ifcompilepart@iv\compilepart@ivfalse

%% Table of contents depth control
\setcounter{tocdepth}{3}

%% Cross-reference formatting
\providecommand{\soiArtRef}[1]{Article~\ref{art:#1}}
\providecommand{\soiPartRef}[1]{Part~\ref{part:#1}}
\providecommand{\soiSchedRef}[1]{Schedule~\ref{sched:#1}}

%% Special constitutional formatting
\providecommand{\soiPreambleFormat}{\centering\large\itshape}
\providecommand{\soiSignatureFormat}{\raggedleft\normalsize}

%% End of configuration}{}}}
    \begin{document}
}{}

\soiArticle{\soiAmendment{Right to education}\soiAmendmentData{Constitution (Eighty-sixth Amendment) Act, 2002, s. 2 \soiWef{1-4-2010}}}{%
    \soiAmendment{The State shall provide free and compulsory education to all children of the age of six to fourteen years in such manner as the State may, by law, determine.}\soiAmendmentData{Constitution (Eighty-sixth Amendment) Act, 2002, s. 2 \soiWef{1-4-2010}}
}

\@ifundefined{documentclass}{}{\end{document}}
% \input{content/articles/article_022}

% Note: Additional chapters would be included here:
% Right against Exploitation (Articles 23-24)
% Right to Freedom of Religion (Articles 25-28)  
% Cultural and Educational Rights (Articles 29-30)
% Right to Constitutional Remedies (Articles 32-35)

% Part IV: Directive Principles of State Policy (selected articles)
%% Part IV: Directive Principles of State Policy
%% Can be compiled independently from any directory level

% Check if this is being compiled as a standalone document
\ifx\documentclass\undefined\else
    \documentclass[a4paper,showamendments]{soi}
    \IfFileExists{config.tex}{%% Configuration file for Source of India LaTeX class
%% This file contains customizable settings and preferences

%% Amendment display preferences
% Uncomment the following to hide amendments globally
% \@soishowamendmentsfalse

%% Typography fine-tuning
\setlength{\parskip}{\baselineskip}
\setlength{\parindent}{0pt}

%% Footnote spacing adjustments
\setlength{\footnotesep}{0.75em}
\addtolength{\skip\footins}{0.25em}

%% Enhanced spacing for constitutional elements
\newlength{\articlespacing}
\setlength{\articlespacing}{1.5\baselineskip}

\newlength{\clausespacing}
\setlength{\clausespacing}{0.75\baselineskip}

%% Custom formatting for special constitutional terms
\providecommand{\soiConstitutionTerm}[1]{\textsc{#1}}
\providecommand{\soiActCitation}[1]{\textit{#1}}

%% Header customization
\providecommand{\soiHeaderFont}{\small\itshape}

%% Page layout fine adjustments
\addtolength{\textheight}{0.5cm}
\addtolength{\textwidth}{0.25cm}

%% Constitutional document metadata
\providecommand{\soiConstitutionTitle}{THE CONSTITUTION OF INDIA}
\providecommand{\soiConstitutionSubtitle}{As amended up to the One Hundred and Fifth Amendment Act, 2021}
\providecommand{\soiConstitutionDate}{26th November, 1949}

%% Amendment citation formatting
\providecommand{\soiAmendmentCite}[2]{%
    Constitution (#1 Amendment) Act, #2%
}

%% Effective date formatting
\providecommand{\soiEffectiveDate}[1]{%
    \textit{w.e.f.} #1%
}

%% Development mode settings
\if@soidraftmode
    % Show additional information in draft mode
    \fancyfoot[LE]{\tiny Draft Mode}
    \fancyfoot[RO]{\tiny \today}
\fi

%% Conditional compilation settings
% Uncomment to compile only specific parts
% \newif\ifcompilepart@i\compilepart@itrue
% \newif\ifcompilepart@iii\compilepart@iiitrue
% \newif\ifcompilepart@iv\compilepart@ivfalse

%% Table of contents depth control
\setcounter{tocdepth}{3}

%% Cross-reference formatting
\providecommand{\soiArtRef}[1]{Article~\ref{art:#1}}
\providecommand{\soiPartRef}[1]{Part~\ref{part:#1}}
\providecommand{\soiSchedRef}[1]{Schedule~\ref{sched:#1}}

%% Special constitutional formatting
\providecommand{\soiPreambleFormat}{\centering\large\itshape}
\providecommand{\soiSignatureFormat}{\raggedleft\normalsize}

%% End of configuration}{%
        \IfFileExists{../config.tex}{%% Configuration file for Source of India LaTeX class
%% This file contains customizable settings and preferences

%% Amendment display preferences
% Uncomment the following to hide amendments globally
% \@soishowamendmentsfalse

%% Typography fine-tuning
\setlength{\parskip}{\baselineskip}
\setlength{\parindent}{0pt}

%% Footnote spacing adjustments
\setlength{\footnotesep}{0.75em}
\addtolength{\skip\footins}{0.25em}

%% Enhanced spacing for constitutional elements
\newlength{\articlespacing}
\setlength{\articlespacing}{1.5\baselineskip}

\newlength{\clausespacing}
\setlength{\clausespacing}{0.75\baselineskip}

%% Custom formatting for special constitutional terms
\providecommand{\soiConstitutionTerm}[1]{\textsc{#1}}
\providecommand{\soiActCitation}[1]{\textit{#1}}

%% Header customization
\providecommand{\soiHeaderFont}{\small\itshape}

%% Page layout fine adjustments
\addtolength{\textheight}{0.5cm}
\addtolength{\textwidth}{0.25cm}

%% Constitutional document metadata
\providecommand{\soiConstitutionTitle}{THE CONSTITUTION OF INDIA}
\providecommand{\soiConstitutionSubtitle}{As amended up to the One Hundred and Fifth Amendment Act, 2021}
\providecommand{\soiConstitutionDate}{26th November, 1949}

%% Amendment citation formatting
\providecommand{\soiAmendmentCite}[2]{%
    Constitution (#1 Amendment) Act, #2%
}

%% Effective date formatting
\providecommand{\soiEffectiveDate}[1]{%
    \textit{w.e.f.} #1%
}

%% Development mode settings
\if@soidraftmode
    % Show additional information in draft mode
    \fancyfoot[LE]{\tiny Draft Mode}
    \fancyfoot[RO]{\tiny \today}
\fi

%% Conditional compilation settings
% Uncomment to compile only specific parts
% \newif\ifcompilepart@i\compilepart@itrue
% \newif\ifcompilepart@iii\compilepart@iiitrue
% \newif\ifcompilepart@iv\compilepart@ivfalse

%% Table of contents depth control
\setcounter{tocdepth}{3}

%% Cross-reference formatting
\providecommand{\soiArtRef}[1]{Article~\ref{art:#1}}
\providecommand{\soiPartRef}[1]{Part~\ref{part:#1}}
\providecommand{\soiSchedRef}[1]{Schedule~\ref{sched:#1}}

%% Special constitutional formatting
\providecommand{\soiPreambleFormat}{\centering\large\itshape}
\providecommand{\soiSignatureFormat}{\raggedleft\normalsize}

%% End of configuration}{%
            \IfFileExists{../../config.tex}{%% Configuration file for Source of India LaTeX class
%% This file contains customizable settings and preferences

%% Amendment display preferences
% Uncomment the following to hide amendments globally
% \@soishowamendmentsfalse

%% Typography fine-tuning
\setlength{\parskip}{\baselineskip}
\setlength{\parindent}{0pt}

%% Footnote spacing adjustments
\setlength{\footnotesep}{0.75em}
\addtolength{\skip\footins}{0.25em}

%% Enhanced spacing for constitutional elements
\newlength{\articlespacing}
\setlength{\articlespacing}{1.5\baselineskip}

\newlength{\clausespacing}
\setlength{\clausespacing}{0.75\baselineskip}

%% Custom formatting for special constitutional terms
\providecommand{\soiConstitutionTerm}[1]{\textsc{#1}}
\providecommand{\soiActCitation}[1]{\textit{#1}}

%% Header customization
\providecommand{\soiHeaderFont}{\small\itshape}

%% Page layout fine adjustments
\addtolength{\textheight}{0.5cm}
\addtolength{\textwidth}{0.25cm}

%% Constitutional document metadata
\providecommand{\soiConstitutionTitle}{THE CONSTITUTION OF INDIA}
\providecommand{\soiConstitutionSubtitle}{As amended up to the One Hundred and Fifth Amendment Act, 2021}
\providecommand{\soiConstitutionDate}{26th November, 1949}

%% Amendment citation formatting
\providecommand{\soiAmendmentCite}[2]{%
    Constitution (#1 Amendment) Act, #2%
}

%% Effective date formatting
\providecommand{\soiEffectiveDate}[1]{%
    \textit{w.e.f.} #1%
}

%% Development mode settings
\if@soidraftmode
    % Show additional information in draft mode
    \fancyfoot[LE]{\tiny Draft Mode}
    \fancyfoot[RO]{\tiny \today}
\fi

%% Conditional compilation settings
% Uncomment to compile only specific parts
% \newif\ifcompilepart@i\compilepart@itrue
% \newif\ifcompilepart@iii\compilepart@iiitrue
% \newif\ifcompilepart@iv\compilepart@ivfalse

%% Table of contents depth control
\setcounter{tocdepth}{3}

%% Cross-reference formatting
\providecommand{\soiArtRef}[1]{Article~\ref{art:#1}}
\providecommand{\soiPartRef}[1]{Part~\ref{part:#1}}
\providecommand{\soiSchedRef}[1]{Schedule~\ref{sched:#1}}

%% Special constitutional formatting
\providecommand{\soiPreambleFormat}{\centering\large\itshape}
\providecommand{\soiSignatureFormat}{\raggedleft\normalsize}

%% End of configuration}{}%
        }%
    }
    \begin{document}
    \def\soistandalone{true}
\fi

\soiPart{IV}{DIRECTIVE PRINCIPLES OF STATE POLICY}

% Path-aware article inclusion
\IfFileExists{../articles/article_045.tex}{%% Article 45: Provision for early childhood care and education to children below the age of six years
%% Can be compiled independently from any directory level

% Path-aware configuration loading
\ifx\documentclass\undefined\else
    \documentclass[a4paper,showamendments]{soi}
    \IfFileExists{config.tex}{%% Configuration file for Source of India LaTeX class
%% This file contains customizable settings and preferences

%% Amendment display preferences
% Uncomment the following to hide amendments globally
% \@soishowamendmentsfalse

%% Typography fine-tuning
\setlength{\parskip}{\baselineskip}
\setlength{\parindent}{0pt}

%% Footnote spacing adjustments
\setlength{\footnotesep}{0.75em}
\addtolength{\skip\footins}{0.25em}

%% Enhanced spacing for constitutional elements
\newlength{\articlespacing}
\setlength{\articlespacing}{1.5\baselineskip}

\newlength{\clausespacing}
\setlength{\clausespacing}{0.75\baselineskip}

%% Custom formatting for special constitutional terms
\providecommand{\soiConstitutionTerm}[1]{\textsc{#1}}
\providecommand{\soiActCitation}[1]{\textit{#1}}

%% Header customization
\providecommand{\soiHeaderFont}{\small\itshape}

%% Page layout fine adjustments
\addtolength{\textheight}{0.5cm}
\addtolength{\textwidth}{0.25cm}

%% Constitutional document metadata
\providecommand{\soiConstitutionTitle}{THE CONSTITUTION OF INDIA}
\providecommand{\soiConstitutionSubtitle}{As amended up to the One Hundred and Fifth Amendment Act, 2021}
\providecommand{\soiConstitutionDate}{26th November, 1949}

%% Amendment citation formatting
\providecommand{\soiAmendmentCite}[2]{%
    Constitution (#1 Amendment) Act, #2%
}

%% Effective date formatting
\providecommand{\soiEffectiveDate}[1]{%
    \textit{w.e.f.} #1%
}

%% Development mode settings
\if@soidraftmode
    % Show additional information in draft mode
    \fancyfoot[LE]{\tiny Draft Mode}
    \fancyfoot[RO]{\tiny \today}
\fi

%% Conditional compilation settings
% Uncomment to compile only specific parts
% \newif\ifcompilepart@i\compilepart@itrue
% \newif\ifcompilepart@iii\compilepart@iiitrue
% \newif\ifcompilepart@iv\compilepart@ivfalse

%% Table of contents depth control
\setcounter{tocdepth}{3}

%% Cross-reference formatting
\providecommand{\soiArtRef}[1]{Article~\ref{art:#1}}
\providecommand{\soiPartRef}[1]{Part~\ref{part:#1}}
\providecommand{\soiSchedRef}[1]{Schedule~\ref{sched:#1}}

%% Special constitutional formatting
\providecommand{\soiPreambleFormat}{\centering\large\itshape}
\providecommand{\soiSignatureFormat}{\raggedleft\normalsize}

%% End of configuration}{\IfFileExists{../config.tex}{%% Configuration file for Source of India LaTeX class
%% This file contains customizable settings and preferences

%% Amendment display preferences
% Uncomment the following to hide amendments globally
% \@soishowamendmentsfalse

%% Typography fine-tuning
\setlength{\parskip}{\baselineskip}
\setlength{\parindent}{0pt}

%% Footnote spacing adjustments
\setlength{\footnotesep}{0.75em}
\addtolength{\skip\footins}{0.25em}

%% Enhanced spacing for constitutional elements
\newlength{\articlespacing}
\setlength{\articlespacing}{1.5\baselineskip}

\newlength{\clausespacing}
\setlength{\clausespacing}{0.75\baselineskip}

%% Custom formatting for special constitutional terms
\providecommand{\soiConstitutionTerm}[1]{\textsc{#1}}
\providecommand{\soiActCitation}[1]{\textit{#1}}

%% Header customization
\providecommand{\soiHeaderFont}{\small\itshape}

%% Page layout fine adjustments
\addtolength{\textheight}{0.5cm}
\addtolength{\textwidth}{0.25cm}

%% Constitutional document metadata
\providecommand{\soiConstitutionTitle}{THE CONSTITUTION OF INDIA}
\providecommand{\soiConstitutionSubtitle}{As amended up to the One Hundred and Fifth Amendment Act, 2021}
\providecommand{\soiConstitutionDate}{26th November, 1949}

%% Amendment citation formatting
\providecommand{\soiAmendmentCite}[2]{%
    Constitution (#1 Amendment) Act, #2%
}

%% Effective date formatting
\providecommand{\soiEffectiveDate}[1]{%
    \textit{w.e.f.} #1%
}

%% Development mode settings
\if@soidraftmode
    % Show additional information in draft mode
    \fancyfoot[LE]{\tiny Draft Mode}
    \fancyfoot[RO]{\tiny \today}
\fi

%% Conditional compilation settings
% Uncomment to compile only specific parts
% \newif\ifcompilepart@i\compilepart@itrue
% \newif\ifcompilepart@iii\compilepart@iiitrue
% \newif\ifcompilepart@iv\compilepart@ivfalse

%% Table of contents depth control
\setcounter{tocdepth}{3}

%% Cross-reference formatting
\providecommand{\soiArtRef}[1]{Article~\ref{art:#1}}
\providecommand{\soiPartRef}[1]{Part~\ref{part:#1}}
\providecommand{\soiSchedRef}[1]{Schedule~\ref{sched:#1}}

%% Special constitutional formatting
\providecommand{\soiPreambleFormat}{\centering\large\itshape}
\providecommand{\soiSignatureFormat}{\raggedleft\normalsize}

%% End of configuration}{\IfFileExists{../../config.tex}{%% Configuration file for Source of India LaTeX class
%% This file contains customizable settings and preferences

%% Amendment display preferences
% Uncomment the following to hide amendments globally
% \@soishowamendmentsfalse

%% Typography fine-tuning
\setlength{\parskip}{\baselineskip}
\setlength{\parindent}{0pt}

%% Footnote spacing adjustments
\setlength{\footnotesep}{0.75em}
\addtolength{\skip\footins}{0.25em}

%% Enhanced spacing for constitutional elements
\newlength{\articlespacing}
\setlength{\articlespacing}{1.5\baselineskip}

\newlength{\clausespacing}
\setlength{\clausespacing}{0.75\baselineskip}

%% Custom formatting for special constitutional terms
\providecommand{\soiConstitutionTerm}[1]{\textsc{#1}}
\providecommand{\soiActCitation}[1]{\textit{#1}}

%% Header customization
\providecommand{\soiHeaderFont}{\small\itshape}

%% Page layout fine adjustments
\addtolength{\textheight}{0.5cm}
\addtolength{\textwidth}{0.25cm}

%% Constitutional document metadata
\providecommand{\soiConstitutionTitle}{THE CONSTITUTION OF INDIA}
\providecommand{\soiConstitutionSubtitle}{As amended up to the One Hundred and Fifth Amendment Act, 2021}
\providecommand{\soiConstitutionDate}{26th November, 1949}

%% Amendment citation formatting
\providecommand{\soiAmendmentCite}[2]{%
    Constitution (#1 Amendment) Act, #2%
}

%% Effective date formatting
\providecommand{\soiEffectiveDate}[1]{%
    \textit{w.e.f.} #1%
}

%% Development mode settings
\if@soidraftmode
    % Show additional information in draft mode
    \fancyfoot[LE]{\tiny Draft Mode}
    \fancyfoot[RO]{\tiny \today}
\fi

%% Conditional compilation settings
% Uncomment to compile only specific parts
% \newif\ifcompilepart@i\compilepart@itrue
% \newif\ifcompilepart@iii\compilepart@iiitrue
% \newif\ifcompilepart@iv\compilepart@ivfalse

%% Table of contents depth control
\setcounter{tocdepth}{3}

%% Cross-reference formatting
\providecommand{\soiArtRef}[1]{Article~\ref{art:#1}}
\providecommand{\soiPartRef}[1]{Part~\ref{part:#1}}
\providecommand{\soiSchedRef}[1]{Schedule~\ref{sched:#1}}

%% Special constitutional formatting
\providecommand{\soiPreambleFormat}{\centering\large\itshape}
\providecommand{\soiSignatureFormat}{\raggedleft\normalsize}

%% End of configuration}{}}}
    \begin{document}
    \def\soistandalone{true}
\fi

\soiArticle{\soiAmendment{Provision for early childhood care and education to children below the age of six years}\soiAmendmentData{Constitution (Eighty-sixth Amendment) Act, 2002, s. 3 \soiWef{1-4-2010}}}{%
    \soiAmendment{The State shall endeavour to provide early childhood care and education for all children until they complete the age of six years.}\soiAmendmentData{Constitution (Eighty-sixth Amendment) Act, 2002, s. 3 \soiWef{1-4-2010}}
}

\ifdefined\soistandalone\end{document}\fi}{%
    \IfFileExists{articles/article_045.tex}{%% Article 45: Provision for early childhood care and education to children below the age of six years
%% Can be compiled independently from any directory level

% Path-aware configuration loading
\ifx\documentclass\undefined\else
    \documentclass[a4paper,showamendments]{soi}
    \IfFileExists{config.tex}{%% Configuration file for Source of India LaTeX class
%% This file contains customizable settings and preferences

%% Amendment display preferences
% Uncomment the following to hide amendments globally
% \@soishowamendmentsfalse

%% Typography fine-tuning
\setlength{\parskip}{\baselineskip}
\setlength{\parindent}{0pt}

%% Footnote spacing adjustments
\setlength{\footnotesep}{0.75em}
\addtolength{\skip\footins}{0.25em}

%% Enhanced spacing for constitutional elements
\newlength{\articlespacing}
\setlength{\articlespacing}{1.5\baselineskip}

\newlength{\clausespacing}
\setlength{\clausespacing}{0.75\baselineskip}

%% Custom formatting for special constitutional terms
\providecommand{\soiConstitutionTerm}[1]{\textsc{#1}}
\providecommand{\soiActCitation}[1]{\textit{#1}}

%% Header customization
\providecommand{\soiHeaderFont}{\small\itshape}

%% Page layout fine adjustments
\addtolength{\textheight}{0.5cm}
\addtolength{\textwidth}{0.25cm}

%% Constitutional document metadata
\providecommand{\soiConstitutionTitle}{THE CONSTITUTION OF INDIA}
\providecommand{\soiConstitutionSubtitle}{As amended up to the One Hundred and Fifth Amendment Act, 2021}
\providecommand{\soiConstitutionDate}{26th November, 1949}

%% Amendment citation formatting
\providecommand{\soiAmendmentCite}[2]{%
    Constitution (#1 Amendment) Act, #2%
}

%% Effective date formatting
\providecommand{\soiEffectiveDate}[1]{%
    \textit{w.e.f.} #1%
}

%% Development mode settings
\if@soidraftmode
    % Show additional information in draft mode
    \fancyfoot[LE]{\tiny Draft Mode}
    \fancyfoot[RO]{\tiny \today}
\fi

%% Conditional compilation settings
% Uncomment to compile only specific parts
% \newif\ifcompilepart@i\compilepart@itrue
% \newif\ifcompilepart@iii\compilepart@iiitrue
% \newif\ifcompilepart@iv\compilepart@ivfalse

%% Table of contents depth control
\setcounter{tocdepth}{3}

%% Cross-reference formatting
\providecommand{\soiArtRef}[1]{Article~\ref{art:#1}}
\providecommand{\soiPartRef}[1]{Part~\ref{part:#1}}
\providecommand{\soiSchedRef}[1]{Schedule~\ref{sched:#1}}

%% Special constitutional formatting
\providecommand{\soiPreambleFormat}{\centering\large\itshape}
\providecommand{\soiSignatureFormat}{\raggedleft\normalsize}

%% End of configuration}{\IfFileExists{../config.tex}{%% Configuration file for Source of India LaTeX class
%% This file contains customizable settings and preferences

%% Amendment display preferences
% Uncomment the following to hide amendments globally
% \@soishowamendmentsfalse

%% Typography fine-tuning
\setlength{\parskip}{\baselineskip}
\setlength{\parindent}{0pt}

%% Footnote spacing adjustments
\setlength{\footnotesep}{0.75em}
\addtolength{\skip\footins}{0.25em}

%% Enhanced spacing for constitutional elements
\newlength{\articlespacing}
\setlength{\articlespacing}{1.5\baselineskip}

\newlength{\clausespacing}
\setlength{\clausespacing}{0.75\baselineskip}

%% Custom formatting for special constitutional terms
\providecommand{\soiConstitutionTerm}[1]{\textsc{#1}}
\providecommand{\soiActCitation}[1]{\textit{#1}}

%% Header customization
\providecommand{\soiHeaderFont}{\small\itshape}

%% Page layout fine adjustments
\addtolength{\textheight}{0.5cm}
\addtolength{\textwidth}{0.25cm}

%% Constitutional document metadata
\providecommand{\soiConstitutionTitle}{THE CONSTITUTION OF INDIA}
\providecommand{\soiConstitutionSubtitle}{As amended up to the One Hundred and Fifth Amendment Act, 2021}
\providecommand{\soiConstitutionDate}{26th November, 1949}

%% Amendment citation formatting
\providecommand{\soiAmendmentCite}[2]{%
    Constitution (#1 Amendment) Act, #2%
}

%% Effective date formatting
\providecommand{\soiEffectiveDate}[1]{%
    \textit{w.e.f.} #1%
}

%% Development mode settings
\if@soidraftmode
    % Show additional information in draft mode
    \fancyfoot[LE]{\tiny Draft Mode}
    \fancyfoot[RO]{\tiny \today}
\fi

%% Conditional compilation settings
% Uncomment to compile only specific parts
% \newif\ifcompilepart@i\compilepart@itrue
% \newif\ifcompilepart@iii\compilepart@iiitrue
% \newif\ifcompilepart@iv\compilepart@ivfalse

%% Table of contents depth control
\setcounter{tocdepth}{3}

%% Cross-reference formatting
\providecommand{\soiArtRef}[1]{Article~\ref{art:#1}}
\providecommand{\soiPartRef}[1]{Part~\ref{part:#1}}
\providecommand{\soiSchedRef}[1]{Schedule~\ref{sched:#1}}

%% Special constitutional formatting
\providecommand{\soiPreambleFormat}{\centering\large\itshape}
\providecommand{\soiSignatureFormat}{\raggedleft\normalsize}

%% End of configuration}{\IfFileExists{../../config.tex}{%% Configuration file for Source of India LaTeX class
%% This file contains customizable settings and preferences

%% Amendment display preferences
% Uncomment the following to hide amendments globally
% \@soishowamendmentsfalse

%% Typography fine-tuning
\setlength{\parskip}{\baselineskip}
\setlength{\parindent}{0pt}

%% Footnote spacing adjustments
\setlength{\footnotesep}{0.75em}
\addtolength{\skip\footins}{0.25em}

%% Enhanced spacing for constitutional elements
\newlength{\articlespacing}
\setlength{\articlespacing}{1.5\baselineskip}

\newlength{\clausespacing}
\setlength{\clausespacing}{0.75\baselineskip}

%% Custom formatting for special constitutional terms
\providecommand{\soiConstitutionTerm}[1]{\textsc{#1}}
\providecommand{\soiActCitation}[1]{\textit{#1}}

%% Header customization
\providecommand{\soiHeaderFont}{\small\itshape}

%% Page layout fine adjustments
\addtolength{\textheight}{0.5cm}
\addtolength{\textwidth}{0.25cm}

%% Constitutional document metadata
\providecommand{\soiConstitutionTitle}{THE CONSTITUTION OF INDIA}
\providecommand{\soiConstitutionSubtitle}{As amended up to the One Hundred and Fifth Amendment Act, 2021}
\providecommand{\soiConstitutionDate}{26th November, 1949}

%% Amendment citation formatting
\providecommand{\soiAmendmentCite}[2]{%
    Constitution (#1 Amendment) Act, #2%
}

%% Effective date formatting
\providecommand{\soiEffectiveDate}[1]{%
    \textit{w.e.f.} #1%
}

%% Development mode settings
\if@soidraftmode
    % Show additional information in draft mode
    \fancyfoot[LE]{\tiny Draft Mode}
    \fancyfoot[RO]{\tiny \today}
\fi

%% Conditional compilation settings
% Uncomment to compile only specific parts
% \newif\ifcompilepart@i\compilepart@itrue
% \newif\ifcompilepart@iii\compilepart@iiitrue
% \newif\ifcompilepart@iv\compilepart@ivfalse

%% Table of contents depth control
\setcounter{tocdepth}{3}

%% Cross-reference formatting
\providecommand{\soiArtRef}[1]{Article~\ref{art:#1}}
\providecommand{\soiPartRef}[1]{Part~\ref{part:#1}}
\providecommand{\soiSchedRef}[1]{Schedule~\ref{sched:#1}}

%% Special constitutional formatting
\providecommand{\soiPreambleFormat}{\centering\large\itshape}
\providecommand{\soiSignatureFormat}{\raggedleft\normalsize}

%% End of configuration}{}}}
    \begin{document}
    \def\soistandalone{true}
\fi

\soiArticle{\soiAmendment{Provision for early childhood care and education to children below the age of six years}\soiAmendmentData{Constitution (Eighty-sixth Amendment) Act, 2002, s. 3 \soiWef{1-4-2010}}}{%
    \soiAmendment{The State shall endeavour to provide early childhood care and education for all children until they complete the age of six years.}\soiAmendmentData{Constitution (Eighty-sixth Amendment) Act, 2002, s. 3 \soiWef{1-4-2010}}
}

\ifdefined\soistandalone\end{document}\fi}{%
        \IfFileExists{content/articles/article_045.tex}{%% Article 45: Provision for early childhood care and education to children below the age of six years
%% Can be compiled independently from any directory level

% Path-aware configuration loading
\ifx\documentclass\undefined\else
    \documentclass[a4paper,showamendments]{soi}
    \IfFileExists{config.tex}{%% Configuration file for Source of India LaTeX class
%% This file contains customizable settings and preferences

%% Amendment display preferences
% Uncomment the following to hide amendments globally
% \@soishowamendmentsfalse

%% Typography fine-tuning
\setlength{\parskip}{\baselineskip}
\setlength{\parindent}{0pt}

%% Footnote spacing adjustments
\setlength{\footnotesep}{0.75em}
\addtolength{\skip\footins}{0.25em}

%% Enhanced spacing for constitutional elements
\newlength{\articlespacing}
\setlength{\articlespacing}{1.5\baselineskip}

\newlength{\clausespacing}
\setlength{\clausespacing}{0.75\baselineskip}

%% Custom formatting for special constitutional terms
\providecommand{\soiConstitutionTerm}[1]{\textsc{#1}}
\providecommand{\soiActCitation}[1]{\textit{#1}}

%% Header customization
\providecommand{\soiHeaderFont}{\small\itshape}

%% Page layout fine adjustments
\addtolength{\textheight}{0.5cm}
\addtolength{\textwidth}{0.25cm}

%% Constitutional document metadata
\providecommand{\soiConstitutionTitle}{THE CONSTITUTION OF INDIA}
\providecommand{\soiConstitutionSubtitle}{As amended up to the One Hundred and Fifth Amendment Act, 2021}
\providecommand{\soiConstitutionDate}{26th November, 1949}

%% Amendment citation formatting
\providecommand{\soiAmendmentCite}[2]{%
    Constitution (#1 Amendment) Act, #2%
}

%% Effective date formatting
\providecommand{\soiEffectiveDate}[1]{%
    \textit{w.e.f.} #1%
}

%% Development mode settings
\if@soidraftmode
    % Show additional information in draft mode
    \fancyfoot[LE]{\tiny Draft Mode}
    \fancyfoot[RO]{\tiny \today}
\fi

%% Conditional compilation settings
% Uncomment to compile only specific parts
% \newif\ifcompilepart@i\compilepart@itrue
% \newif\ifcompilepart@iii\compilepart@iiitrue
% \newif\ifcompilepart@iv\compilepart@ivfalse

%% Table of contents depth control
\setcounter{tocdepth}{3}

%% Cross-reference formatting
\providecommand{\soiArtRef}[1]{Article~\ref{art:#1}}
\providecommand{\soiPartRef}[1]{Part~\ref{part:#1}}
\providecommand{\soiSchedRef}[1]{Schedule~\ref{sched:#1}}

%% Special constitutional formatting
\providecommand{\soiPreambleFormat}{\centering\large\itshape}
\providecommand{\soiSignatureFormat}{\raggedleft\normalsize}

%% End of configuration}{\IfFileExists{../config.tex}{%% Configuration file for Source of India LaTeX class
%% This file contains customizable settings and preferences

%% Amendment display preferences
% Uncomment the following to hide amendments globally
% \@soishowamendmentsfalse

%% Typography fine-tuning
\setlength{\parskip}{\baselineskip}
\setlength{\parindent}{0pt}

%% Footnote spacing adjustments
\setlength{\footnotesep}{0.75em}
\addtolength{\skip\footins}{0.25em}

%% Enhanced spacing for constitutional elements
\newlength{\articlespacing}
\setlength{\articlespacing}{1.5\baselineskip}

\newlength{\clausespacing}
\setlength{\clausespacing}{0.75\baselineskip}

%% Custom formatting for special constitutional terms
\providecommand{\soiConstitutionTerm}[1]{\textsc{#1}}
\providecommand{\soiActCitation}[1]{\textit{#1}}

%% Header customization
\providecommand{\soiHeaderFont}{\small\itshape}

%% Page layout fine adjustments
\addtolength{\textheight}{0.5cm}
\addtolength{\textwidth}{0.25cm}

%% Constitutional document metadata
\providecommand{\soiConstitutionTitle}{THE CONSTITUTION OF INDIA}
\providecommand{\soiConstitutionSubtitle}{As amended up to the One Hundred and Fifth Amendment Act, 2021}
\providecommand{\soiConstitutionDate}{26th November, 1949}

%% Amendment citation formatting
\providecommand{\soiAmendmentCite}[2]{%
    Constitution (#1 Amendment) Act, #2%
}

%% Effective date formatting
\providecommand{\soiEffectiveDate}[1]{%
    \textit{w.e.f.} #1%
}

%% Development mode settings
\if@soidraftmode
    % Show additional information in draft mode
    \fancyfoot[LE]{\tiny Draft Mode}
    \fancyfoot[RO]{\tiny \today}
\fi

%% Conditional compilation settings
% Uncomment to compile only specific parts
% \newif\ifcompilepart@i\compilepart@itrue
% \newif\ifcompilepart@iii\compilepart@iiitrue
% \newif\ifcompilepart@iv\compilepart@ivfalse

%% Table of contents depth control
\setcounter{tocdepth}{3}

%% Cross-reference formatting
\providecommand{\soiArtRef}[1]{Article~\ref{art:#1}}
\providecommand{\soiPartRef}[1]{Part~\ref{part:#1}}
\providecommand{\soiSchedRef}[1]{Schedule~\ref{sched:#1}}

%% Special constitutional formatting
\providecommand{\soiPreambleFormat}{\centering\large\itshape}
\providecommand{\soiSignatureFormat}{\raggedleft\normalsize}

%% End of configuration}{\IfFileExists{../../config.tex}{%% Configuration file for Source of India LaTeX class
%% This file contains customizable settings and preferences

%% Amendment display preferences
% Uncomment the following to hide amendments globally
% \@soishowamendmentsfalse

%% Typography fine-tuning
\setlength{\parskip}{\baselineskip}
\setlength{\parindent}{0pt}

%% Footnote spacing adjustments
\setlength{\footnotesep}{0.75em}
\addtolength{\skip\footins}{0.25em}

%% Enhanced spacing for constitutional elements
\newlength{\articlespacing}
\setlength{\articlespacing}{1.5\baselineskip}

\newlength{\clausespacing}
\setlength{\clausespacing}{0.75\baselineskip}

%% Custom formatting for special constitutional terms
\providecommand{\soiConstitutionTerm}[1]{\textsc{#1}}
\providecommand{\soiActCitation}[1]{\textit{#1}}

%% Header customization
\providecommand{\soiHeaderFont}{\small\itshape}

%% Page layout fine adjustments
\addtolength{\textheight}{0.5cm}
\addtolength{\textwidth}{0.25cm}

%% Constitutional document metadata
\providecommand{\soiConstitutionTitle}{THE CONSTITUTION OF INDIA}
\providecommand{\soiConstitutionSubtitle}{As amended up to the One Hundred and Fifth Amendment Act, 2021}
\providecommand{\soiConstitutionDate}{26th November, 1949}

%% Amendment citation formatting
\providecommand{\soiAmendmentCite}[2]{%
    Constitution (#1 Amendment) Act, #2%
}

%% Effective date formatting
\providecommand{\soiEffectiveDate}[1]{%
    \textit{w.e.f.} #1%
}

%% Development mode settings
\if@soidraftmode
    % Show additional information in draft mode
    \fancyfoot[LE]{\tiny Draft Mode}
    \fancyfoot[RO]{\tiny \today}
\fi

%% Conditional compilation settings
% Uncomment to compile only specific parts
% \newif\ifcompilepart@i\compilepart@itrue
% \newif\ifcompilepart@iii\compilepart@iiitrue
% \newif\ifcompilepart@iv\compilepart@ivfalse

%% Table of contents depth control
\setcounter{tocdepth}{3}

%% Cross-reference formatting
\providecommand{\soiArtRef}[1]{Article~\ref{art:#1}}
\providecommand{\soiPartRef}[1]{Part~\ref{part:#1}}
\providecommand{\soiSchedRef}[1]{Schedule~\ref{sched:#1}}

%% Special constitutional formatting
\providecommand{\soiPreambleFormat}{\centering\large\itshape}
\providecommand{\soiSignatureFormat}{\raggedleft\normalsize}

%% End of configuration}{}}}
    \begin{document}
    \def\soistandalone{true}
\fi

\soiArticle{\soiAmendment{Provision for early childhood care and education to children below the age of six years}\soiAmendmentData{Constitution (Eighty-sixth Amendment) Act, 2002, s. 3 \soiWef{1-4-2010}}}{%
    \soiAmendment{The State shall endeavour to provide early childhood care and education for all children until they complete the age of six years.}\soiAmendmentData{Constitution (Eighty-sixth Amendment) Act, 2002, s. 3 \soiWef{1-4-2010}}
}

\ifdefined\soistandalone\end{document}\fi}{%
            \ClassWarning{soi}{Cannot find article_045.tex}%
        }%
    }%
}

% Note: Additional articles would be included here
% Articles 36-44, 46-51 etc.

\ifdefined\soistandalone\end{document}\fi

% Note: Additional parts would be included as work progresses
% Part II: Citizenship
% Part V: The Union
% Part VI: The States
% Part VII: [Repealed]
% Part VIII: The Union Territories
% Part IX: The Panchayats
% Part IXA: The Municipalities
% Part IXB: The Co-operative Societies
% Part X: The Scheduled and Tribal Areas
% Part XI: Relations between the Union and the States
% Part XII: Finance, Property, Contracts and Suits
% Part XIII: Trade, Commerce and intercourse within the territory of India
% Part XIV: Services under the Union and the States
% Part XIVA: Tribunals
% Part XV: Elections
% Part XVI: Special Provisions Relating to Certain Classes
% Part XVII: Official Language
% Part XVIII: Emergency Provisions
% Part XIX: Miscellaneous
% Part XX: Amendment of the Constitution
% Part XXI: Temporary, Transitional and Special Provisions
% Part XXII: Short Title, Commencement, Authoritative Text in Hindi and Repeals

\clearpage

% Selected Schedules
%% First Schedule: The States
%% Can be compiled independently from any directory level

% Check if this is being compiled as a standalone document
\ifx\documentclass\undefined\else
    \documentclass[a4paper,showamendments]{soi}
    \IfFileExists{config.tex}{%% Configuration file for Source of India LaTeX class
%% This file contains customizable settings and preferences

%% Amendment display preferences
% Uncomment the following to hide amendments globally
% \@soishowamendmentsfalse

%% Typography fine-tuning
\setlength{\parskip}{\baselineskip}
\setlength{\parindent}{0pt}

%% Footnote spacing adjustments
\setlength{\footnotesep}{0.75em}
\addtolength{\skip\footins}{0.25em}

%% Enhanced spacing for constitutional elements
\newlength{\articlespacing}
\setlength{\articlespacing}{1.5\baselineskip}

\newlength{\clausespacing}
\setlength{\clausespacing}{0.75\baselineskip}

%% Custom formatting for special constitutional terms
\providecommand{\soiConstitutionTerm}[1]{\textsc{#1}}
\providecommand{\soiActCitation}[1]{\textit{#1}}

%% Header customization
\providecommand{\soiHeaderFont}{\small\itshape}

%% Page layout fine adjustments
\addtolength{\textheight}{0.5cm}
\addtolength{\textwidth}{0.25cm}

%% Constitutional document metadata
\providecommand{\soiConstitutionTitle}{THE CONSTITUTION OF INDIA}
\providecommand{\soiConstitutionSubtitle}{As amended up to the One Hundred and Fifth Amendment Act, 2021}
\providecommand{\soiConstitutionDate}{26th November, 1949}

%% Amendment citation formatting
\providecommand{\soiAmendmentCite}[2]{%
    Constitution (#1 Amendment) Act, #2%
}

%% Effective date formatting
\providecommand{\soiEffectiveDate}[1]{%
    \textit{w.e.f.} #1%
}

%% Development mode settings
\if@soidraftmode
    % Show additional information in draft mode
    \fancyfoot[LE]{\tiny Draft Mode}
    \fancyfoot[RO]{\tiny \today}
\fi

%% Conditional compilation settings
% Uncomment to compile only specific parts
% \newif\ifcompilepart@i\compilepart@itrue
% \newif\ifcompilepart@iii\compilepart@iiitrue
% \newif\ifcompilepart@iv\compilepart@ivfalse

%% Table of contents depth control
\setcounter{tocdepth}{3}

%% Cross-reference formatting
\providecommand{\soiArtRef}[1]{Article~\ref{art:#1}}
\providecommand{\soiPartRef}[1]{Part~\ref{part:#1}}
\providecommand{\soiSchedRef}[1]{Schedule~\ref{sched:#1}}

%% Special constitutional formatting
\providecommand{\soiPreambleFormat}{\centering\large\itshape}
\providecommand{\soiSignatureFormat}{\raggedleft\normalsize}

%% End of configuration}{%
        \IfFileExists{../config.tex}{%% Configuration file for Source of India LaTeX class
%% This file contains customizable settings and preferences

%% Amendment display preferences
% Uncomment the following to hide amendments globally
% \@soishowamendmentsfalse

%% Typography fine-tuning
\setlength{\parskip}{\baselineskip}
\setlength{\parindent}{0pt}

%% Footnote spacing adjustments
\setlength{\footnotesep}{0.75em}
\addtolength{\skip\footins}{0.25em}

%% Enhanced spacing for constitutional elements
\newlength{\articlespacing}
\setlength{\articlespacing}{1.5\baselineskip}

\newlength{\clausespacing}
\setlength{\clausespacing}{0.75\baselineskip}

%% Custom formatting for special constitutional terms
\providecommand{\soiConstitutionTerm}[1]{\textsc{#1}}
\providecommand{\soiActCitation}[1]{\textit{#1}}

%% Header customization
\providecommand{\soiHeaderFont}{\small\itshape}

%% Page layout fine adjustments
\addtolength{\textheight}{0.5cm}
\addtolength{\textwidth}{0.25cm}

%% Constitutional document metadata
\providecommand{\soiConstitutionTitle}{THE CONSTITUTION OF INDIA}
\providecommand{\soiConstitutionSubtitle}{As amended up to the One Hundred and Fifth Amendment Act, 2021}
\providecommand{\soiConstitutionDate}{26th November, 1949}

%% Amendment citation formatting
\providecommand{\soiAmendmentCite}[2]{%
    Constitution (#1 Amendment) Act, #2%
}

%% Effective date formatting
\providecommand{\soiEffectiveDate}[1]{%
    \textit{w.e.f.} #1%
}

%% Development mode settings
\if@soidraftmode
    % Show additional information in draft mode
    \fancyfoot[LE]{\tiny Draft Mode}
    \fancyfoot[RO]{\tiny \today}
\fi

%% Conditional compilation settings
% Uncomment to compile only specific parts
% \newif\ifcompilepart@i\compilepart@itrue
% \newif\ifcompilepart@iii\compilepart@iiitrue
% \newif\ifcompilepart@iv\compilepart@ivfalse

%% Table of contents depth control
\setcounter{tocdepth}{3}

%% Cross-reference formatting
\providecommand{\soiArtRef}[1]{Article~\ref{art:#1}}
\providecommand{\soiPartRef}[1]{Part~\ref{part:#1}}
\providecommand{\soiSchedRef}[1]{Schedule~\ref{sched:#1}}

%% Special constitutional formatting
\providecommand{\soiPreambleFormat}{\centering\large\itshape}
\providecommand{\soiSignatureFormat}{\raggedleft\normalsize}

%% End of configuration}{%
            \IfFileExists{../../config.tex}{%% Configuration file for Source of India LaTeX class
%% This file contains customizable settings and preferences

%% Amendment display preferences
% Uncomment the following to hide amendments globally
% \@soishowamendmentsfalse

%% Typography fine-tuning
\setlength{\parskip}{\baselineskip}
\setlength{\parindent}{0pt}

%% Footnote spacing adjustments
\setlength{\footnotesep}{0.75em}
\addtolength{\skip\footins}{0.25em}

%% Enhanced spacing for constitutional elements
\newlength{\articlespacing}
\setlength{\articlespacing}{1.5\baselineskip}

\newlength{\clausespacing}
\setlength{\clausespacing}{0.75\baselineskip}

%% Custom formatting for special constitutional terms
\providecommand{\soiConstitutionTerm}[1]{\textsc{#1}}
\providecommand{\soiActCitation}[1]{\textit{#1}}

%% Header customization
\providecommand{\soiHeaderFont}{\small\itshape}

%% Page layout fine adjustments
\addtolength{\textheight}{0.5cm}
\addtolength{\textwidth}{0.25cm}

%% Constitutional document metadata
\providecommand{\soiConstitutionTitle}{THE CONSTITUTION OF INDIA}
\providecommand{\soiConstitutionSubtitle}{As amended up to the One Hundred and Fifth Amendment Act, 2021}
\providecommand{\soiConstitutionDate}{26th November, 1949}

%% Amendment citation formatting
\providecommand{\soiAmendmentCite}[2]{%
    Constitution (#1 Amendment) Act, #2%
}

%% Effective date formatting
\providecommand{\soiEffectiveDate}[1]{%
    \textit{w.e.f.} #1%
}

%% Development mode settings
\if@soidraftmode
    % Show additional information in draft mode
    \fancyfoot[LE]{\tiny Draft Mode}
    \fancyfoot[RO]{\tiny \today}
\fi

%% Conditional compilation settings
% Uncomment to compile only specific parts
% \newif\ifcompilepart@i\compilepart@itrue
% \newif\ifcompilepart@iii\compilepart@iiitrue
% \newif\ifcompilepart@iv\compilepart@ivfalse

%% Table of contents depth control
\setcounter{tocdepth}{3}

%% Cross-reference formatting
\providecommand{\soiArtRef}[1]{Article~\ref{art:#1}}
\providecommand{\soiPartRef}[1]{Part~\ref{part:#1}}
\providecommand{\soiSchedRef}[1]{Schedule~\ref{sched:#1}}

%% Special constitutional formatting
\providecommand{\soiPreambleFormat}{\centering\large\itshape}
\providecommand{\soiSignatureFormat}{\raggedleft\normalsize}

%% End of configuration}{}%
        }%
    }
    \begin{document}
    \def\soistandalone{true}
\fi

\soiSchedule{FIRST}{THE STATES}{%

\textbf{I. THE STATES}

\begin{enumerate}
    \item Andhra Pradesh.—The territories specified in sub-section (1) of section 3 of the Andhra State Act, 1953, \soiAmendment{sub-section (1) of section 3 of the States Reorganisation Act, 1956, the First Schedule to the Andhra Pradesh and Madras (Alteration of Boundaries) Act, 1959, and the First Schedule to the Andhra Pradesh and Mysore (Transfer of Territory) Act, 1968, but excluding the territories specified in the Second Schedule to the Andhra Pradesh and Madras (Alteration of Boundaries) Act, 1959}\soiAmendmentData{Constitution (Seventh Amendment) Act, 1956, Sch. \soiWef{1-11-1956}}.

    \item \soiAmendment{Assam.—The territories which immediately before the commencement of this Constitution were comprised in the Province of Assam, the Khasi States and the Assam Tribal Areas, but excluding the territories specified in the Schedule to the Assam (Alteration of Boundaries) Act, 1951}\soiAmendmentData{Constitution (One Hundred and First Amendment) Act, 2016, s. 16 \soiWef{16-9-2016}}.

    \item Bihar.—The territories which immediately before the commencement of this Constitution were \soiAmendment{either comprised in the Province of Bihar or were being administered as if they formed part of that Province}\soiAmendmentData{Constitution (Twelfth Amendment) Act, 1962, s. 3 \soiWef{20-12-1961}}, \soiAmendment{but excluding the territories specified in sub-section (1) of section 3 of the Bihar Reorganisation Act, 2000}\soiAmendmentData{Bihar Reorganisation Act, 2000 \soiWef{15-11-2000}}.

    % Additional states would continue here...
    
\end{enumerate}

\textbf{II. THE UNION TERRITORIES}

\begin{enumerate}
    \item \soiAmendment{Delhi.—The territory which immediately before the commencement of this Constitution was comprised in the Chief Commissioner's Province of Delhi}\soiAmendmentData{Constitution (Seventh Amendment) Act, 1956, Sch. \soiWef{1-11-1956}}.

    \item \soiAmendment{The Andaman and Nicobar Islands.—The territory which immediately before the commencement of this Constitution was comprised in the Chief Commissioner's Province of the Andaman and Nicobar Islands}\soiAmendmentData{Constitution (Seventh Amendment) Act, 1956, Sch. \soiWef{1-11-1956}}.

    \item \soiAmendment{Lakshadweep.—The territory which immediately before the commencement of this Constitution was comprised in the Chief Commissioner's Province of the Laccadive, Minicoy and Amindivi Islands}\soiAmendmentData{Constitution (Tenth Amendment) Act, 1961, s. 2 \soiWef{16-8-1961}}.

    % Additional union territories...

\end{enumerate}

}

\ifdefined\soistandalone\end{document}\fi
\input{content/schedules/schedule_12}

% Note: Additional schedules would be included as work progresses
% Second Schedule: Provisions as to the President and the Governors
% Third Schedule: Forms of Oaths or Affirmations
% Fourth Schedule: Allocation of seats in the Council of States
% Fifth Schedule: Provisions as to the Administration and Control of Scheduled Areas and Scheduled Tribes
% Sixth Schedule: Provisions as to the Administration of Tribal Areas in Assam, Meghalaya, Tripura and Mizoram
% Seventh Schedule: Distribution of Legislative Powers
% Eighth Schedule: Languages
% Ninth Schedule: Validation of certain Acts and Regulations
% Tenth Schedule: Provisions as to Disqualification on ground of defection
% Eleventh Schedule: Powers, authority and responsibilities of Panchayats

\end{document}