%% Main document for Constitution of India
%% Using Source of India LaTeX Class v1.1 (Fixed Version)
%% This is the main working document - NOT a sample

\documentclass[a4paper,12pt,showamendments,twoside]{soi}

%% Load enhanced configuration
%% Configuration file for Source of India LaTeX class
%% This file contains customizable settings and preferences

%% Amendment display preferences
% Uncomment the following to hide amendments globally
% \@soishowamendmentsfalse

%% Typography fine-tuning
\setlength{\parskip}{\baselineskip}
\setlength{\parindent}{0pt}

%% Footnote spacing adjustments
\setlength{\footnotesep}{0.75em}
\addtolength{\skip\footins}{0.25em}

%% Enhanced spacing for constitutional elements
\newlength{\articlespacing}
\setlength{\articlespacing}{1.5\baselineskip}

\newlength{\clausespacing}
\setlength{\clausespacing}{0.75\baselineskip}

%% Custom formatting for special constitutional terms
\providecommand{\soiConstitutionTerm}[1]{\textsc{#1}}
\providecommand{\soiActCitation}[1]{\textit{#1}}

%% Header customization
\providecommand{\soiHeaderFont}{\small\itshape}

%% Page layout fine adjustments
\addtolength{\textheight}{0.5cm}
\addtolength{\textwidth}{0.25cm}

%% Constitutional document metadata
\providecommand{\soiConstitutionTitle}{THE CONSTITUTION OF INDIA}
\providecommand{\soiConstitutionSubtitle}{As amended up to the One Hundred and Fifth Amendment Act, 2021}
\providecommand{\soiConstitutionDate}{26th November, 1949}

%% Amendment citation formatting
\providecommand{\soiAmendmentCite}[2]{%
    Constitution (#1 Amendment) Act, #2%
}

%% Effective date formatting
\providecommand{\soiEffectiveDate}[1]{%
    \textit{w.e.f.} #1%
}

%% Development mode settings
\if@soidraftmode
    % Show additional information in draft mode
    \fancyfoot[LE]{\tiny Draft Mode}
    \fancyfoot[RO]{\tiny \today}
\fi

%% Conditional compilation settings
% Uncomment to compile only specific parts
% \newif\ifcompilepart@i\compilepart@itrue
% \newif\ifcompilepart@iii\compilepart@iiitrue
% \newif\ifcompilepart@iv\compilepart@ivfalse

%% Table of contents depth control
\setcounter{tocdepth}{3}

%% Cross-reference formatting
\providecommand{\soiArtRef}[1]{Article~\ref{art:#1}}
\providecommand{\soiPartRef}[1]{Part~\ref{part:#1}}
\providecommand{\soiSchedRef}[1]{Schedule~\ref{sched:#1}}

%% Special constitutional formatting
\providecommand{\soiPreambleFormat}{\centering\large\itshape}
\providecommand{\soiSignatureFormat}{\raggedleft\normalsize}

%% End of configuration

%% Document metadata
\title{The Constitution of India}
\author{Government of India}
\date{26th November, 1949}

\begin{document}

%% Front matter
\frontmatter
\pagenumbering{roman}

%% Title page
\begin{titlepage}
    \centering
    \vspace*{2cm}
    
    {\Huge\bfseries THE CONSTITUTION\\[0.5cm]OF INDIA}
    
    \vspace{2cm}
    
    {\Large As Amended}
    
    \vspace{1cm}
    
    {\large (Showing all Constitutional Amendments)}
    
    \vspace{3cm}
    
    % \includegraphics[width=4cm]{emblem}% Ashoka Chakra emblem (if available)
    
    \vfill
    
    {\large GOVERNMENT OF INDIA\\
    MINISTRY OF LAW AND JUSTICE\\
    (LEGISLATIVE DEPARTMENT)}
    
    \vspace{1cm}
    
    {\large NEW DELHI}
    
    \vspace{0.5cm}
    
    {\normalsize \today}
\end{titlepage}

%% Table of Contents
\tableofcontents
\cleardoublepage

%% Preamble using fixed modular file
%% Preamble to the Constitution of India
%% This is the opening statement of the Constitution

\phantomsection
\addcontentsline{toc}{part}{Preamble}

\begin{center}
{\LARGE \textbf{PREAMBLE}}
\end{center}

\vspace{1cm}

\begin{center}
\begin{minipage}{0.8\textwidth}
\centering
\large

WE, THE PEOPLE OF INDIA, having solemnly resolved to constitute India into a 
\Amendment{SOVEREIGN SOCIALIST SECULAR DEMOCRATIC REPUBLIC}{The words ``SOCIALIST SECULAR'' were 
inserted by the Constitution (Forty-second Amendment) Act, 1976, s. 2 (w.e.f. 3-1-1977)} 
and to secure to all its citizens:

\vspace{0.5cm}

JUSTICE, social, economic and political;

\vspace{0.3cm}

LIBERTY of thought, expression, belief, faith and worship;

\vspace{0.3cm}

EQUALITY of status and of opportunity;

and to promote among them all

\vspace{0.3cm}

FRATERNITY assuring the dignity of the individual and the 
\Amendment{unity and integrity of the Nation}{Substituted for ``unity of the Nation'' 
by the Constitution (Forty-second Amendment) Act, 1976, s. 2 (w.e.f. 3-1-1977)};

\vspace{0.5cm}

IN OUR CONSTITUENT ASSEMBLY this twenty-sixth day of November, 1949, do HEREBY ADOPT, ENACT AND GIVE TO OURSELVES THIS CONSTITUTION.

\end{minipage}
\end{center}

\vspace{2cm}

% Optional: Add signature section for historical reference
\begin{flushright}
\textit{Dr. Rajendra Prasad} \\
\textit{President, Constituent Assembly}
\end{flushright}

\cleardoublepage

%% Main content - switch to Arabic numbering
\mainmatter
\pagenumbering{arabic}

%% Include all Parts using modular structure
\ifcompilepart
    %% PART I - THE UNION AND ITS TERRITORY
    %% PART I - THE UNION AND ITS TERRITORY

\Part{I}{THE UNION AND ITS TERRITORY}

%% Include individual article files
%% Article 1 - Name and territory of the Union
\DeclareArticle{1}{}{union-territory}
\Article{Name and territory of the Union}{%
    \Clause{India, that is Bharat, shall be a Union of States.}
    
    \Clause{The States and the territories thereof shall be as specified in the First Schedule.}
    
    \Clause{The territory of India shall comprise---}
    
    \SubClause{the territories of the States;}
    
    \SubClause{the Union territories specified in the First Schedule; and}
    
    \SubClause{such other territories as may be acquired.}
}
\input{content/articles/article_2}
%% Article 3 - Formation of new States and alteration of areas, boundaries or names of existing States
\DeclareArticle{3}{}{union-territory}
\Article{Formation of new States and alteration of areas, boundaries or names of existing States}{%
    Parliament may by law---
    
    \SubClause{form a new State by separation of territory from any State or by uniting two or more States or parts of States or by uniting any territory to a part of any State;}
    
    \SubClause{increase the area of any State;}
    
    \SubClause{diminish the area of any State;}
    
    \SubClause{alter the boundaries of any State;}
    
    \SubClause{alter the name of any State:}
    
    \Proviso{no Bill for the purpose shall be introduced in either House of Parliament except on the recommendation of the President and unless, where the proposal contained in the Bill affects the area, boundaries or name of any of the States, the Bill has been referred by the President to the Legislature of that State for expressing its views thereon within such period as may be specified in the reference or within such further period as the President may allow and the period so specified or allowed has expired.}
    
    \Explanation{In this article, in clauses (a) to (e), "State" includes a Union territory, but in the proviso, "State" does not include a Union territory.}
}
    
    %% PART II - CITIZENSHIP  
    %% PART II - CITIZENSHIP

\Part{II}{CITIZENSHIP}

%% Include individual article files for citizenship
%% Article 5 - Citizenship at the commencement of the Constitution
\DeclareArticle{5}{}{citizenship}
\Article{Citizenship at the commencement of the Constitution}{%
    At the commencement of this Constitution, every person who has his domicile in the territory of India and---
    
    \SubClause{who was born in the territory of India; or}
    
    \SubClause{either of whose parents was born in the territory of India; or}
    
    \SubClause{who has been ordinarily resident in the territory of India for not less than five years immediately preceding such commencement,}
    
    shall be a citizen of India.
}
%% Article 6 - Rights of citizenship of certain persons who have migrated to India from Pakistan
\DeclareArticle{6}{}{citizenship}
\Article{Rights of citizenship of certain persons who have migrated to India from Pakistan}{%
    Notwithstanding anything in article 5, a person who has migrated to the territory of India from the territory now included in Pakistan shall be deemed to be a citizen of India at the commencement of this Constitution if---
    
    \SubClause{either of his parents or any of his grand-parents was born in India as defined in the Government of India Act, 1935 (as originally enacted); and}
    
    \SubClause{\Clause{he has been ordinarily resident in the territory of India since the date of his migration, or}}
    
    \SubClause{\Clause{such person or either of his parents or any of his grand-parents was born in India as defined in the Government of India Act, 1935 (as originally enacted), and he has been registered as a citizen of India by an officer appointed in that behalf by the Government of the Dominion of India on an application made by him therefor to such officer before the commencement of this Constitution in the form and manner prescribed by that Government:}}
    
    \Proviso{no person shall be so deemed to be a citizen of India if he has voluntarily acquired the citizenship of any foreign State.}
}
\input{content/articles/article_7}
    
    %% PART III - FUNDAMENTAL RIGHTS (with fixed constitutional sections)
    %% PART III - FUNDAMENTAL RIGHTS

\Part{III}{FUNDAMENTAL RIGHTS}

\section*{General}
\input{content/articles/article_12}
%% Article 13 - Laws inconsistent with or in derogation of the fundamental rights
\DeclareArticle{13}{}{fundamental-rights}
\Article{Laws inconsistent with or in derogation of the fundamental rights}{%
    \Clause{All laws in force in the territory of India immediately before the commencement of this Constitution, in so far as they are inconsistent with the provisions of this Part, shall, to the extent of such inconsistency, be void.}
    
    \Clause{The State shall not make any law which takes away or abridges the rights conferred by this Part and any law made in contravention of this clause shall, to the extent of the contravention, be void.}
    
    \Clause{In this article, unless the context otherwise requires,---}
    
    \SubClause{"law" includes any Ordinance, order, bye-law, rule, regulation, notification, custom or usage having in the territory of India the force of law;}
    
    \SubClause{"laws in force" includes laws passed or made by a Legislature or other competent authority in the territory of India before the commencement of this Constitution and not previously repealed, notwithstanding that any such law or any part thereof may not be then in operation either at all or in particular areas.}
    
    \Clause{\Inserted{Nothing in this article shall apply to any amendment of this Constitution made under article 368.}{Constitution (Twenty-fifth Amendment) Act, 1971, s. 3 \wef{20-4-1972}}}
}

\section*{Right to Equality}
%% Article 14 - Equality before law
%% Part of original Constitution, no amendments

\DeclareArticle{14}{}{}

\Article{Equality before law}{%
    The State shall not deny to any person equality before the law or the equal protection of the laws within the territory of India.
}
%% Article 15 - Prohibition of discrimination
\DeclareArticle{15}{}{fundamental-rights}
\Article{Prohibition of discrimination on grounds of religion, race, caste, sex or place of birth}{%
    \Clause{The State shall not discriminate against any citizen on grounds only of religion, race, caste, sex, place of birth or any of them.}
    
    \Clause{No citizen shall, on grounds only of religion, race, caste, sex, place of birth or any of them, be subject to any disability, liability, restriction or condition with regard to---}
    
    \SubClause{access to shops, public restaurants, hotels and places of public entertainment; or}
    
    \SubClause{the use of wells, tanks, bathing ghats, roads and places of public resort maintained wholly or partly out of State funds or dedicated to the use of the general public.}
    
    \Clause{Nothing in this article shall prevent the State from making any special provision for women and children.}
    
    \InsertedClause{Nothing in this article or in clause (2) of article 29 shall prevent the State from making any special provision for the advancement of any socially and educationally backward classes of citizens or for the Scheduled Castes and the Scheduled Tribes.}{Constitution (First Amendment) Act, 1951, s. 2 \wef{18-6-1951}}
}

\section*{Right to Freedom}
%% Article 19 - Protection of certain rights regarding freedom of speech, etc.
\DeclareArticle{19}{}{fundamental-rights}
\Article{Protection of certain rights regarding freedom of speech, etc.}{%
    \Clause{All citizens shall have the right---}
    
    \SubClause{to freedom of speech and expression;}
    
    \SubClause{to assemble peaceably and without arms;}
    
    \SubClause{to form associations or unions \Inserted{or co-operative societies}{Constitution (Ninety-seventh Amendment) Act, 2011, s. 2 \wef{12-1-2012}};}
    
    \SubClause{to move freely throughout the territory of India;}
    
    \SubClause{to reside and settle in any part of the territory of India; and}
    
    \SubClause{\Substituted{to practise any profession, or to carry on any occupation, trade or business.}{to acquire, hold and dispose of property; and to practise any profession, or to carry on any occupation, trade or business.}{Constitution (Forty-fourth Amendment) Act, 1978, s. 2 \wef{20-6-1979}}}
    
    \Clause{Nothing in sub-clause (a) of clause (1) shall affect the operation of any existing law, or prevent the State from making any law, in so far as such law imposes reasonable restrictions on the exercise of the right conferred by the said sub-clause in the interests of \Inserted{the sovereignty and integrity of India,}{Constitution (First Amendment) Act, 1951, s. 3 \wef{18-6-1951}} the security of the State, friendly relations with foreign States, public order, decency or morality, or in relation to contempt of court, defamation or incitement to an offence.}
}

\section*{Right to Life and Personal Liberty}
%% Article 21 - Protection of life and personal liberty
\DeclareArticle{21}{}{fundamental-rights}
\Article{Protection of life and personal liberty}{%
    No person shall be deprived of his life or personal liberty except according to procedure established by law.
}
\input{content/articles/article_21a}
    
    %% TODO: Add remaining parts as they are completed
    % \input{content/parts/part_iv}     % Directive Principles of State Policy
    % \input{content/parts/part_iva}    % Fundamental Duties
    % \input{content/parts/part_v}      % The Union
    % \input{content/parts/part_vi}     % The States
    % \input{content/parts/part_vii}    % [Repealed]
    % \input{content/parts/part_viii}   % The Union Territories
    % \input{content/parts/part_ix}     % The Panchayats
    % \input{content/parts/part_ixa}    % The Municipalities
    % \input{content/parts/part_ixb}    % The Co-operative Societies
    % \input{content/parts/part_x}      % The Scheduled and Tribal Areas
    % \input{content/parts/part_xi}     % Relations between the Union and the States
    % \input{content/parts/part_xii}    % Finance, Property, Contracts and Suits
    % \input{content/parts/part_xiii}   % Trade, Commerce and Intercourse
    % \input{content/parts/part_xiv}    % Services Under the Union and the States
    % \input{content/parts/part_xiva}   % Tribunals
    % \input{content/parts/part_xv}     % Elections
    % \input{content/parts/part_xvi}    % Special Provisions for Certain Classes
    % \input{content/parts/part_xvii}   % Official Language
    % \input{content/parts/part_xviii}  % Emergency Provisions
    % \input{content/parts/part_xix}    % Miscellaneous
    % \input{content/parts/part_xx}     % Amendment of the Constitution
    % \input{content/parts/part_xxi}    % Temporary, Transitional and Special Provisions
    % \input{content/parts/part_xxii}   % Short Title, Commencement, Authoritative Text in Hindi and Repeals
\fi

%% Include all Schedules using modular structure
\ifcompileschedules
    \backmatter
    \cleardoublepage
    \appendix
    
    %% All Schedules
    %% First Schedule - The States
\DeclareSchedule{I}{states}
\Schedule{FIRST}{THE STATES}{%
    \section*{I. THE STATES}
    
    \begin{longtable}{p{0.1\textwidth}p{0.4\textwidth}p{0.4\textwidth}}
    \toprule
    \textbf{No.} & \textbf{Name} & \textbf{Territories} \\
    \midrule
    \endhead
    
    1. & Andhra Pradesh & The territories specified in sub-section (1) of section 3 of the Andhra Pradesh Reorganisation Act, 2014. \\
    
    2. & \Substituted{Assam}{Assam}{Constitution (One Hundred and First Amendment) Act, 2016, s. 15 \wef{16-9-2016}} & The territories which immediately before the commencement of this Constitution were comprised in the Province of Assam, the Khasi States and the Assam Tribal Areas. \\
    
    3. & Bihar & The territories which immediately before the commencement of this Constitution were comprised in the Province of Bihar. \\
    
    4. & Gujarat & The territories specified in sub-section (1) of section 3 of the Bombay Reorganisation Act, 1960. \\
    
    5. & Haryana & The territories specified in sub-section (1) of section 3 of the Punjab Reorganisation Act, 1966. \\
    
    6. & Himachal Pradesh & The territories specified in sub-section (1) of section 3 of the State of Himachal Pradesh Act, 1970. \\
    
    7. & \Inserted{Jharkhand}{Constitution (Eighty-eighth Amendment) Act, 2003, s. 2 \wef{15-11-2000}} & \Inserted{The territories specified in sub-section (1) of section 3 of the Bihar Reorganisation Act, 2000.}{Constitution (Eighty-eighth Amendment) Act, 2003, s. 2 \wef{15-11-2000}} \\
    
    8. & Karnataka & The territories specified in sub-section (1) of section 3 of the States Reorganisation Act, 1956. \\
    
    9. & Kerala & The territories specified in sub-section (1) of section 5 of the States Reorganisation Act, 1956. \\
    
    10. & Madhya Pradesh & The territories specified in sub-section (1) of section 9 of the States Reorganisation Act, 1956. \\
    
    \bottomrule
    \end{longtable}
    
    \section*{II. THE UNION TERRITORIES}
    
    \begin{longtable}{p{0.1\textwidth}p{0.4\textwidth}p{0.4\textwidth}}
    \toprule
    \textbf{No.} & \textbf{Name} & \textbf{Territories} \\
    \midrule
    \endhead
    
    1. & \Substituted{Andaman and Nicobar Islands}{Andaman and Nicobar Islands}{Constitution (One Hundred and First Amendment) Act, 2016, s. 15 \wef{16-9-2016}} & The territory which immediately before the commencement of this Constitution was known as the Andaman and Nicobar Islands. \\
    
    2. & Chandigarh & The territory specified in section 4 of the Punjab Reorganisation Act, 1966. \\
    
    3. & \Inserted{Dadra and Nagar Haveli and Daman and Diu}{Dadra and Nagar Haveli and Daman and Diu (Merger of Union Territories) Act, 2019, s. 3 \wef{26-1-2020}} & \Inserted{The territories which immediately before the commencement of the Dadra and Nagar Haveli and Daman and Diu (Merger of Union Territories) Act, 2019 were the Union territories of Dadra and Nagar Haveli and Daman and Diu.}{Dadra and Nagar Haveli and Daman and Diu (Merger of Union Territories) Act, 2019, s. 3 \wef{26-1-2020}} \\
    
    4. & Delhi & The territory which immediately before the commencement of this Constitution was comprised in the Chief Commissioner's Province of Delhi. \\
    
    5. & \Inserted{Jammu and Kashmir}{Constitution (One Hundred and Third Amendment) Act, 2019, s. 2 \wef{31-10-2019}} & \Inserted{The territory which, immediately before the commencement of the Constitution (Application to Jammu and Kashmir) Order, 2019, was included in the State of Jammu and Kashmir but excluding the territories specified in the Second Schedule to the Jammu and Kashmir Reorganisation Act, 2019.}{Constitution (One Hundred and Third Amendment) Act, 2019, s. 2 \wef{31-10-2019}} \\
    
    6. & \Inserted{Ladakh}{Constitution (One Hundred and Third Amendment) Act, 2019, s. 2 \wef{31-10-2019}} & \Inserted{The territories specified in the Second Schedule to the Jammu and Kashmir Reorganisation Act, 2019.}{Constitution (One Hundred and Third Amendment) Act, 2019, s. 2 \wef{31-10-2019}} \\
    
    7. & Lakshadweep & The territory specified in section 6 of the States Reorganisation Act, 1956. \\
    
    8. & Puducherry & The territories which immediately before the commencement of the Constitution (Fourteenth Amendment) Act, 1962, were comprised in the Union territory of Pondicherry. \\
    
    \bottomrule
    \end{longtable}
}    % The States
    
    %% TODO: Add remaining schedules as they are completed
    % \input{content/schedules/schedule_ii}   % [Provisions as to the President and the Governors]
    % \input{content/schedules/schedule_iii}  % Forms of Oaths or Affirmations
    % \input{content/schedules/schedule_iv}   % Allocation of seats in the Council of States
    % \input{content/schedules/schedule_v}    % Provisions as to the Administration and Control of Scheduled Areas and Scheduled Tribes
    % \input{content/schedules/schedule_vi}   % Provisions as to the Administration of Tribal Areas in the States of Assam, Meghalaya, Tripura and Mizoram
    % \input{content/schedules/schedule_vii}  % Union, State and Concurrent Lists
    % \input{content/schedules/schedule_viii} % Languages
    % \input{content/schedules/schedule_ix}   % Validation of certain Acts and Regulations
    % \input{content/schedules/schedule_x}    % [Provisions as to Disqualification on ground of defection]
    % \input{content/schedules/schedule_xi}   % Powers, authority and responsibilities of Panchayats
    % \input{content/schedules/schedule_xii}  % Powers, authority and responsibilities of Municipalities
\fi

%% Document completion status and notes
\ifcompilepart
    \cleardoublepage
    \section*{Document Status}
    
    \subsection*{Compilation Status}
    This document is a work in progress using the enhanced SOI LaTeX Class v1.1. Current status:
    
    \subsubsection*{Completed Parts}
    \begin{itemize}
    \item Part I: The Union and Its Territory (Articles 1-4)
    \item Part II: Citizenship (Articles 5-11)  
    \item Part III: Fundamental Rights (Articles 12-35) - Partially completed with fixed amendment system
    \end{itemize}
    
    \subsubsection*{Completed Schedules}
    \begin{itemize}
    \item First Schedule: The States (Partially completed with fixed amendment syntax)
    \end{itemize}
    
    \subsubsection*{Recent Improvements (v1.1)}
    \begin{itemize}
    \item \textbf{Fixed Amendment System}: Amendment text now appears in main document with citations in footnotes
    \item \textbf{Two-sided Layout}: Proper binding offset and outer margin headers
    \item \textbf{Enhanced Typography}: Constitutional sections properly sized and positioned
    \item \textbf{Improved Footnotes}: Better spacing above and below separator line
    \item \textbf{Page Management}: Section headings stay with their articles
    \end{itemize}
    
    \subsubsection*{Remaining Work}
    \begin{itemize}
    \item Parts IV through XXII (19 parts remaining)
    \item Schedules II through XII (11 schedules remaining) 
    \item Complete article content for existing parts
    \item Full amendment tracking and citations using fixed syntax
    \item Cross-references and indexing
    \end{itemize}
    
    \subsection*{Usage Notes}
    \begin{itemize}
    \item Use \texttt{showamendments} class option to display corrected amendment footnotes
    \item Use \texttt{hideamendments} class option to hide amendment information
    \item Use \texttt{twoside} class option for proper two-sided printing (recommended)
    \item Set \texttt{\textbackslash{}compilepartfalse} in config.tex to skip parts during compilation
    \item Set \texttt{\textbackslash{}compileschedulesfalse} in config.tex to skip schedules during compilation
    \end{itemize}
    
    \subsection*{Amendment System Usage (Fixed in v1.1)}
    \begin{itemize}
    \item \texttt{\textbackslash{}Inserted\{text\}\{citation\}} - Inserts text in main document, citation in footnote
    \item \texttt{\textbackslash{}Substituted\{new\}\{old\}\{citation\}} - Shows new text in main, old text and citation in footnote
    \item \texttt{\textbackslash{}InsertedClause\{text\}\{citation\}} - For inserted clauses with proper numbering
    \item \texttt{\textbackslash{}Omitted\{citation\}} - Shows omitted content when \texttt{showomitted} option is used
    \item \texttt{\textbackslash{}ConstitutionalSection\{title\}} - For section headings like "Right to Equality"
    \end{itemize}
    
    \subsection*{Contributing}
    To add new articles, parts, or schedules using the fixed system:
    \begin{enumerate}
    \item Create the appropriate .tex file in the correct directory
    \item Use the \texttt{\textbackslash{}DeclareArticle}, \texttt{\textbackslash{}Article}, \texttt{\textbackslash{}Part}, or \texttt{\textbackslash{}Schedule} commands
    \item Include the file in the appropriate part file using \texttt{\textbackslash{}input}
    \item Add amendment information using the corrected commands with proper parameter order
    \item Use \texttt{\textbackslash{}ConstitutionalSection} for major section headings within parts
    \end{enumerate}
    
    \subsection*{Compilation}
    \begin{itemize}
    \item Use \texttt{pdflatex main.tex} (run twice for cross-references)
    \item Or use the enhanced compilation script: \texttt{tools/compile\_fixed.sh}
    \item For cleanup: \texttt{tools/cleanup.sh}
    \end{itemize}
    
    \subsection*{Document Features}
    \begin{itemize}
    \item Two-sided layout optimized for binding
    \item Article names on outer margins (away from binding)
    \item Proper footnote positioning and spacing
    \item Constitutional section headings with appropriate typography
    \item Enhanced page break management
    \item Corrected amendment display system
    \end{itemize}
\fi

\end{document}