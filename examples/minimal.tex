%% Minimal working example for the Source of India LaTeX class
%% This demonstrates basic usage with a few articles

\documentclass[a4paper,12pt,showamendments]{soi}

\begin{document}

% Simple title page
\begin{titlepage}
\centering
\vspace*{3cm}
{\Huge \textbf{Constitution of India}} \\[1cm]
{\Large Minimal Example} \\[2cm]
{\large Demonstrating SoI LaTeX Class}
\vfill
{\large \today}
\end{titlepage}

% Simple table of contents
\tableofcontents
\newpage

% Preamble (simplified)
\phantomsection
\addcontentsline{toc}{part}{Preamble}
\begin{center}
{\LARGE \textbf{PREAMBLE}}
\end{center}

\vspace{1cm}

WE, THE PEOPLE OF INDIA, having solemnly resolved to constitute India into a SOVEREIGN DEMOCRATIC REPUBLIC...

\newpage

% Part III - Fundamental Rights (sample)
\Part{III}{FUNDAMENTAL RIGHTS}

\SubGroup{Right to Equality}

% Article 14
\DeclareArticle{14}{}{}
\Article{Equality before law}{%
    The State shall not deny to any person equality before the law or the equal protection of the laws within the territory of India.
}

% Article 21A with amendment
\DeclareArticle{21}{A}{}
\Article{Right to education}{%
    \Amendment{The State shall provide free and compulsory education to all children of the age of six to fourteen years in such manner as the State may, by law, determine.}{Inserted by the Constitution (Eighty-sixth Amendment) Act, 2002, s. 2 (w.e.f. 1-4-2010)}%
}

% Example of article with clauses
\DeclareArticle{19}{}{}
\Article{Protection of certain rights regarding freedom of speech, etc.}{%
    All citizens shall have the right---
    
    \Clause{to freedom of speech and expression;}
    
    \Clause{to assemble peaceably and without arms;}
    
    \Clause{to form associations or unions;}
    
    \Clause{to move freely throughout the territory of India;}
    
    \Clause{to reside and settle in any part of the territory of India;}
    
    \Clause{to practise any profession, or to carry on any occupation, trade or business.}
}

% Example schedule
\Schedule{I}{THE STATES}

\SchedulePart{A}{THE STATES}

\begin{enumerate}
\item Andhra Pradesh
\item Assam  
\item Bihar
\item Gujarat
\item ... (other states)
\end{enumerate}

\end{document}